\section{Campo Gravitatorio}

\subsection{Leyes de Kepler}

Las leyes de Kepler son 3 leyes acerca de las \textbf{órbitas} de los \textbf{planetas} alrededor del Sol, que se deducen a partir de la ley de gravitación universal.

\subsubsection{Primera Ley de Kepler}

Los planetas describen \textbf{órbitas elípticas} alrededor del Sol. El sol está situado en uno de los \textbf{focos} de la elipse.

\begin{figure}[H]
	\centering
	\includegraphics[scale=0.2]{GRAV_1LEYKEPLER.jpg}
	\caption{Primera Ley de Kepler}
	\label{GRAV_1LEYKEPLER}
\end{figure}

\subsubsection{Segunda Ley de Kepler}

Los planetas giran con una \textbf{velocidad areolar constante}, es decir, el vector posición (radiovector) barre áreas iguales en tiempos iguales:

\[
\frac{dA}{dt} = \textnormal{cte.}
\]

Esto quiere decir que la velocidad en el \textbf{perihelio} (punto más cercano al Sol) es \textbf{mayor} que la velocidad en el \textbf{afelio} (punto más lejano al Sol).

\begin{figure}[H]
	\centering
	\includegraphics[scale=0.3]{GRAV_2LEYKEPLER.jpeg}
	\caption{Segunda Ley de Kepler}
	\label{GRAV_2KEPLER}
\end{figure}

\subsubsection{Tercera Ley de Kepler}

El \textbf{cuadrado} de los \textbf{periodos} ($T$) alrededor del Sol es \textbf{proporcional} al \textbf{cubo} de los \textbf{radios} medios de sus órbitas ($r$). Es decir:

\begin{equation}\label{GRAV_3LEYKEPLER}
	T^2 = k \cdot r^3
\end{equation}

Siendo k una \textbf{constante} igual para \textbf{todos los planetas}.

\subsection{Demostración de la Tercera Ley de Kepler}


\begin{equation}\label{GRAV_3KEPLER}
	\begin{rcases}
		\begin{aligned}
			G\frac{Mm}{r^2} = m\frac{v^2}{r} \Longrightarrow v^2= \frac{GM}{r}  \\
			T=\frac{2\pi r}{v}\Longrightarrow v^2=\frac{4\pi^2r^2}{T^2}
		\end{aligned}
	\end{rcases}
	\frac{GM}{r} = \frac{4\pi^2r^2}{T^2} \Longrightarrow T^2 = \frac{4\pi^2}{GM}r^3
\end{equation}



\subsection{El Momento Angular de los Planetas}

Cuando una partícula describe un movimiento curvilíneo, su estado de \textbf{movimiento} se caracteriza por su \textbf{momento angular} o momento cinético ($\vec{L}$):


\begin{equation}\label{GRAV_MOMENTOANGULAR}
	\vec{L} = \vec{r} \times \vec{p} = \vec{r} \times m\vec{v} \Longrightarrow {L} = r\cdot m\cdot v \cdot \sin{\alpha}
\end{equation}

\subsubsection{Teorema de Conservación del Momento Angular}

Fórmulas utilizadas: $\vec{M} = \vec{r}\times \vec{F}$, $\vec{L} = \vec{r} \times m\vec{v}$, $\vec{v} = \frac{d\vec{r}}{dt}$, $\vec{a} = \frac{d\vec{v}}{dt}$

\begin{equation}\label{GRAV_TEOREMACONSERVACIONL}
	\frac{d\vec{L}}{dt} = \frac{d}{dt} \left( \vec{r}\times m\vec{v}\right) = m \left(\cancelto{\vec{v}\times \vec{v} = 0}{\frac{d\vec{r}}{dt} \times \vec{v}} + \vec{r} \times \frac{d\vec{v}}{dt}\right) = \vec{r} \times \vec{F} = \vec{M}
\end{equation}

Como $\frac{d \vec{L}}{dt} = \vec{M}$, si $\vec{M} = 0 \Longrightarrow \vec{L} = \textnormal{cte.}$


\subsection{Ley de Gravitación Universal}

Conocida como la ley gravitacional de Newton, se expresa el \textbf{valor} de la \textbf{fuerza de atracción} entre dos masas. La \textbf{dirección} del vector es la recta que une las dos partículas. Las \textbf{fuerzas} que interactúan entre dos masas tienen el mismo \textbf{módulo} y \textbf{dirección} pero distinto sentido.


\begin{equation}\label{GRAV_LEYGRAVUNIVERSAL}
	\vec{F} = -G \frac{Mm}{r^2} \vec{u_r}
\end{equation}

Donde la Constante de Gravitación Universal es $G = \qty{6.67e-11}{\newton\meter\squared\per\kilo\gram\squared}$

\subsection{Campo Gravitatorio}

El campo gravitatorio se define como la \textbf{perturbación} que la masa produce en el \textbf{espacio} que le \textbf{rodea} por el hecho de tener masa.

\begin{enumerate}
	\item \textsc{\textbf{Líneas de Campo Gravitatorio}}:
	\begin{enumerate}
		\item Son \textbf{radiales} y van \textbf{dirigidos a la masa}.
		\item Son \textbf{tangentes} en cada punto al vector intensidad de campo y tienen su mismo sentido.
		\item No tienen \textbf{origen definido} (ya que el alcance del campo gravitatorio es infinito), pero terminan en puntos materiales denominados \textbf{sumideros} de campo.
		\item La \textbf{densidad} de líneas de campo es \textbf{proporcional} al módulo de la \textbf{intensidad} del campo.
		\item Las líneas de campo \textbf{no se pueden cortar}, ya que eso significaría que en un punto del espacio, el campo tendría dos valores distintos.
	\end{enumerate}
	\item \textsc{\textbf{Intensidad de Campo Gravitatorio ($\vec{g}$)}}
	
	\begin{equation}\label{GRAV_CAMPOGRAVITATORIO}
		\vec{g} = -G \frac{M}{r^2}\vec{u_r} \Longrightarrow g = G\frac{M}{r^2}
	\end{equation}
	
	
	\item \textsc{\textbf{Principio de Superposición}}
	\begin{enumerate}
		\item La intensidad del campo gravitatorio en un punto es la \textbf{suma vectorial} de los campos que crearía cada cuerpo aislado.
		\item De igual forma, la fuerza gravitatoria que siente una masa debido a otras masas será la \textbf{suma} de las fuerzas que cada una ejerzan:
		\begin{equation}\label{GRAV_SUPERPOS}
			\vec{F_T} = \Sigma \vec{F_i}
		\end{equation}
	\end{enumerate}
\end{enumerate}

\begin{figure}[H]
	\centering
	\includegraphics[scale=2]{GRAV_CAMPOGRAV}
	\caption{Líneas de Campo Gravitatorio}
	\label{GRAV_CAMPOGRAV}
\end{figure}

\subsection{Fuerzas Conservativas y Energías }

Las \textbf{fuerzas} son \textbf{conservativas} cuando el trabajo que realiza dicha fuerza para trasladar una partícula de un punto A a otro B depende de los puntos inicial y final, pero \textbf{no} del camino \textbf{seguido}. En este caso, la \textbf{gravedad} es una fuerza \textbf{conservativa}.

\subsubsection{Energía Potencial Gravitatoria}

La energía potencial gravitatoria ($E_p$) es aquella que posee una masa $m$ por \textbf{encontrarse} bajo la \textbf{influencia} gravitatoria de otra masa $M$ u otras masas. También puede definirse como el \textbf{trabajo} que realizaría el campo gravitatorio para \textbf{trasladar} una masa m desde un punto hasta el \textbf{infinito}.

\begin{equation}\label{GRAV_ENERGIAPOTENCIAL}
	E_p = -G \frac{Mm}{r}
\end{equation}

\begin{figure}[H]
	\centering
	\includegraphics[scale=0.5]{GRAV_EPG}
	\caption{Gráfica de la Energía Potencial Gravitatoria}
	\label{GRAV_EPG}
\end{figure}

\subsection{¿Por qué la \texorpdfstring{$E_p$}{Ep} también es \texorpdfstring{$m\cdot g\cdot h$}{mgh}?}

Con esta expresión, se asume que $h\ll R_t$

\begin{equation}
	\begin{split}
		\Delta E_p &= E_{pB} - E_{pA} = -G\frac{Mm}{R_t+h} + G\frac{Mm}{R_t} \\
		&= GMm \left(\frac{1}{R_t}-\frac{1}{R_t+h}\right) = \frac{GMmh}{R_t(R_t+h\simeq R_t)}
	\end{split}
\end{equation}

\begin{equation} 
	\Delta E_p = \frac{GM}{R^2_t}mh = mgh
\end{equation}


\subsubsection{Potencial Gravitatorio}

El potencial gravitatorio (V) en un punto se define como la \textbf{energía potencial gravitatoria} por \textbf{unidad} de masa en dicho punto:

\begin{equation}\label{GRAV_POTENCIAL}
	E_p = m\cdot V \Longrightarrow V = -G\frac{M}{r}
\end{equation}

\subsubsection{Trabajo}

Si $W > 0$, el trabajo lo realiza el \textbf{campo gravitatorio}. Si $W < 0$ el trabajo lo realiza una \textbf{fuerza exterior al campo}. Si $W = 0$ \textbf{no} se realiza \textbf{trabajo}. El trabajo para llevar una partícula de masa m desde un punto A hasta uno B será:

\begin{enumerate}
		\item $W_{A\rightarrow B} = -\Delta E_p$
		\item $W_{A\rightarrow B} = -m\Delta V$
\end{enumerate}

\subsection{Satélites}

\subsubsection{Velocidad Orbital}

Es la \textbf{velocidad necesaria} para que una masa $m$ (como un satélite, tanto natural como artificial) describa una \textbf{órbita circular} alrededor de \textbf{otra} $M$, para lo que la fuerza \textbf{centrípeta} debe ser \textbf{igual} que la fuerza \textbf{gravitatoria}.

\begin{equation}\label{GRAV_VORB}
	G\frac{M\cancel{m}}{r^{\cancel{2}}} = \cancel{m}\frac{v^2}{\cancel{r}} \Longrightarrow v_{\textnormal{orbital}} = \sqrt{\frac{GM}{r}}
\end{equation}

\subsubsection{Satélites Geoestacionarios}

Se llaman \textbf{satélites geosíncronos} a aquellos satélites cuyo periodo de revolución coincide con el de la Tierra: $T = 24\textnormal{h} = 86400\textnormal{s}$. Si además estos satélites están todo el rato \textbf{sobre el mismo punto} de la superficie terrestre (para lo que es necesario que su plano orbital coincida con el ecuador) entonces se denominan \textbf{satélites geoestacionarios}.

\subsubsection{Velocidad de Escape}

La velocidad de escape es la velocidad mínima que debe adquirir un cuerpo para escapar de la \textbf{atracción gravitatoria del planeta} en cuyas \textbf{proximidades} se encuentre. Esto significa que $E_p = 0$, $E_c = 0$ y $E_m = 0$.

\begin{equation}\label{GRAV_VESC}
	v_\textnormal{escape} = \sqrt{\frac{2GM}{r}} = \sqrt{2}\cdot v_\textnormal{orbital}
\end{equation}



\subsubsection{Energía Mecánica de un Satélite en Órbita}

La energía mecánica de un satélite en órbita, también denominada energía orbital, es la \textbf{suma} de la energía \textbf{potencial} y \textbf{cinética}:

\begin{equation}\label{GRAV_ENERGIAMECANICASATELITE}
	E_m = -G\frac{Mm}{2r}
\end{equation}

\subsubsection{Principio de Conservación de la Energía}

Es la energía que debemos \textbf{comunicarle} a un satélite para que \textbf{pase} de una \textbf{órbita} a otra, donde las energías mecánicas son distintas.

\begin{equation}\label{GRAV_CONSERVACIONENERGIA}
	W_{\textnormal{com}} + E_{p_A} + E_{c_A} = E_{c_B} + E_{p_B}
\end{equation}

De esta fórmula se puede \textbf{deducir} la fórmula de la \textbf{\textsc{energía de satelización}} (energía necesaria para poner un satélite en órbita):

\begin{equation}\label{GRAV_ENERGIASATELIZACION}
	E_{\textnormal{satelización}} = GMm \left( \frac{1}{r_A} - \frac{1}{2\cdot r_B} \right)
\end{equation}

\begin{mybox}{Conceptos Clave: Gravitación}
	\begin{itemize}
		\item \textbf{Distancia $r$:} En todas las fórmulas, $r$ es la distancia al \textbf{centro} del planeta. Si te dan la altura $h$, recuerda: $r = R_T + h$.
		\item \textbf{Signos:} La Energía Potencial ($E_p$) y el Potencial ($V$) son siempre \textbf{negativos} (el 0 está en el infinito).
	\end{itemize}
\end{mybox}



\subsection{Resumen de Fórmulas}
	
\begin{table}[H]
	\centering
	\renewcommand{\arraystretch}{2}
	\setlength{\tabcolsep}{6pt}
	\setlength{\aboverulesep}{4pt}
	\setlength{\belowrulesep}{4pt}
	\renewcommand{\tabularxcolumn}[1]{m{#1}}
	\rowcolors{2}{gray!10}{white}
	\begin{tabularx}{\textwidth}{
			>{\centering\arraybackslash}m{4.5cm}
			>{\centering\arraybackslash}X
			>{\centering\arraybackslash}m{3cm}}
		\toprule
		\rowcolor{gray!25}
		\textbf{Magnitud / Concepto} & \textbf{Fórmula} & \textbf{Unidades (SI)} \\
		\midrule
		
		\textsc{Ley de Gravitación Universal} & $\displaystyle \vec{F} = -G \frac{Mm}{r^2}\vec{u_r}$ & \unit{\newton} \\
		
		\textsc{Intensidad del Campo Gravitatorio} & $\displaystyle \vec{g} = -G\frac{M}{r^2}\vec{u_r}$ & \unit{\meter\per\second\squared} \\
		
		\textsc{Cte. Grav. Universal} & $\displaystyle G = \num{6.67e-11}$ & \unit{\newton\meter\squared\per\kilo\gram\squared} \\
		
		\textsc{Momento Angular} & $\displaystyle \vec{L} = \vec{r}\times m\vec{v}$ & \unit{\kilo\gram\meter\squared\per\second} \\
		
		\textsc{Energía Potencial Gravitatoria} & $\displaystyle E_p = -G\frac{Mm}{r} = mV$ & \unit{\joule} \\
		
		\textsc{Potencial Gravitatorio} & $\displaystyle V = -G\frac{M}{r}$ & \unit{\joule\per\kilo\gram} \\
		
		\textsc{Trabajo del Campo Gravitatorio} & $\displaystyle W_{A\rightarrow B} = -\Delta E_p = -m\Delta V$ & \unit{\joule} \\
		
		\textsc{Velocidad Orbital} & $\displaystyle v_{\textnormal{orbital}} = \sqrt{\frac{GM}{r}}$ & \unit{\meter\per\second} \\
		
		\textsc{Velocidad de Escape} & $\displaystyle v_{\textnormal{escape}} = \sqrt{\frac{2GM}{r}}$ & \unit{\meter\per\second} \\
		
		\textsc{Energía Mecánica en Órbita} & $\displaystyle E_m = -G\frac{Mm}{2r}$ & \unit{\joule} \\
		
		\textsc{Energía de Satelización} & $GMm\left(\frac{1}{r_A} - \frac{1}{2r_B}\right)$ & \unit{\joule} \\
		
		\textsc{Periodo Orbital} & $\displaystyle T = \frac{2\pi r}{v}$ & \unit{\second} \\
		
		\bottomrule
	\end{tabularx}
	\label{GRAV_TABLAFORMULARIO}
	\caption{Formulario de Gravedad}
\end{table}
