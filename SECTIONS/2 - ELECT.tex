\section{Campo Eléctrico}

\subsection{Cargas Puntuales}

\subsubsection{Ley de Coulomb}

La fuerza de \textbf{atracción o repulsión} entre 2 cargas puntuales es directamente proporcional al producto de las cargas e \textbf{inversamente proporcional} al cuadrado de la distancia que las separa.

\begin{equation}\label{ELEC_COULOUMB}
	\vec{F} = k\frac{Qq}{r^2}\vec{u_r} \Longrightarrow F=k\frac{Qq}{r^2}
\end{equation}

Donde $k = \SI{9e9}{\newton\meter\squared\per\coulomb\squared}$ en el vacío, que está relacionada con $\varepsilon$ (permitividad eléctrica):

\[k = \frac{1}{4\pi\varepsilon}\]

\subsubsection{Líneas de Campo Eléctrico}

El campo eléctrico es la \textbf{perturbación} que genera un cuerpo por tener carga eléctrica.
\begin{enumerate}
	\item Si la carga es \textbf{positiva}, el campo eléctrico es de \textbf{repulsión}.
	\item Si la carga es \textbf{negativa}, el campo eléctrico es de \textbf{atracción}.
\end{enumerate}

El campo eléctrico, igual que el campo gravitatorio, es un \textbf{campo conservativo}.

\begin{figure}[H]
	\centering
	\includegraphics[scale=0.6]{ELEC_LINEAS}
	\caption{Líneas de Campo Magnético}
	\label{ELEC_LINEAS}
\end{figure}

\subsubsection{Intensidad del Campo Eléctrico}

La \textbf{intensidad} de campo eléctrico ($\vec{E}$) (también llamada simplemente campo eléctrico) en un punto se define como la \textbf{fuerza} que se ejerce por \textbf{unidad} de \textbf{carga} positiva situada en dicho punto:

\begin{equation} \label{ELEC_FCAMPOELECTRICO}
	\vec{F} = q\cdot\vec{E} 
\end{equation}

\begin{equation} \label{ELEC_CAMPOELECTRICO}
	\vec{E} = k\frac{Q}{r^2}\cdot\vec{u_r} \Longrightarrow E = k \frac{Q}{r^2}
\end{equation}


\subsubsection{Principio de Superposición}

El Principio de Superposición establece que el efecto total de varias causas actuando juntas es la suma de los efectos que cada causa produciría por separado.

\begin{equation}\label{ELEC_SUPERPOS}
	\vec{F_T} = \Sigma F_i
\end{equation}

\begin{equation}\label{ELEC_DISTANCIA}
	\Delta V = -\vec{E} \cdot \Delta\vec{r}
\end{equation}

\subsection{Energías y Fuerzas Conservativas}

La energía potencial eléctrica ($E_p$) es aquella que \textbf{posee} una carga por \textbf{encontrarse} bajo la influencia \textbf{eléctrica} de otra carga u otras cargas:

\begin{equation}\label{ELEC_EPOT}
	E_p = k\frac{Qq}{r}
\end{equation}

\subsubsection{Potencial Eléctrico (\texorpdfstring{$V$}{V})}

Es el trabajo \textbf{realizado} por el campo eléctrico para \textbf{trasladar} una \textbf{unidad} de carga desde un punto hasta el \textbf{infinito}.

\begin{equation}\label{ELEC_POENCIAL}
	V = \frac{E_p}{q} \Longrightarrow V = k\frac{Q}{r}
\end{equation}

\subsubsection{Trabajo (\texorpdfstring{$W$}{W})}

La fuerza \textbf{eléctrica} (al igual que la fuerza gravitatoria) es una \textbf{fuerza central}, ya que está dirigida hacia un punto. El campo eléctrico (igual que el campo gravitatorio) es un \textbf{campo conservativo}. El trabajo realizado por el campo \textbf{depende} solo de su estado \textbf{inicial} y \textbf{final}, no depende de la \textbf{trayectoria}.

\begin{equation}\label{ELEC_TRABAJO}
	W_{A\rightarrow B} = E_{pA} - E_{pB} \Longrightarrow W_{A\rightarrow B} = -q(V_B - V_A)
\end{equation}

\clearpage

\subsection{Campos Eléctricos Uniformes}

\subsubsection{Líneas de Campo}

\begin{wrapfigure}{R}{0.5\textwidth}
	\centering
	\includegraphics[scale=0.15]{ELEC_LINEASCAMPOSUNIFORMES}
	\caption{Líneas de Campo Eléctrico en un Campo Uniforme}
	\label{ELEC_LINEASCAMPOUNIFORME}
\end{wrapfigure}

Las líneas de campo de un campo eléctrico \textbf{uniforme} son \textbf{líneas} de campo \textbf{paralelas}.

\subsubsection{Movimiento de Carga en Campo Uniforme}

Cuando trabajamos con un campo eléctrico uniforme, al ser \textbf{constante}, aplicaremos también la \textbf{segunda ley de Newton}, $\Sigma F = ma$, siendo esta \textbf{aceleración} también \textbf{uniforme}.

Por lo tanto, podremos utilizar en este caso las \textbf{fórmulas del MRUA}:

\begin{equation}\label{ELEC_MOVUNIFORME}
	\begin{rcases}
		\begin{aligned}
			x = x_0 + v_0 t + \frac{1}{2}at^2  \\
			v = v_0 + at \\
			v^2 = v_o^2 + 2a\Delta x
		\end{aligned}
	\end{rcases}
\end{equation}

Podemos resolverlo también mediante el \textbf{Principio de Conservación de la Energía Mecánica}, al ser la fuerza eléctrica \textbf{conservativa}:

\begin{equation}\label{ELEC_CONSERVACIONENERGIA}
	\Delta E_m = 0 \Longrightarrow \Delta E_c = -\Delta E_p
\end{equation}

\subsubsection{Campo: Láminas Infinitas}

\begin{figure}[H]
	\centering
	\includegraphics[scale=0.05]{ELEC_CAMPOSLAMINASINFINITAS}
	\caption{Campo Eléctrico Creado por Láminas Infinitas}
	\label{ELEC_CAMPOLAMINASINFINITAS}
\end{figure}

\begin{equation}\label{ELEC_LAMINASINFINITAS}
	E = \frac{\sigma}{\varepsilon}
\end{equation}

Siendo $\sigma$ las cargas por superficie $\frac{\textnormal{cargas ($C$)}}{\textnormal{superficie ($m^2$)}}$ y $\varepsilon$ depende del material.

\subsubsection{Campo: Hilo Infinito}

Las líneas de campo salen radialmente del hilo (suponiendo $\lambda$ positivo).

\begin{figure}[H]
	\centering
	\includegraphics[scale=0.3]{ELEC_CAMPOSHILOINFINITO}
	\caption{Campo Eléctrico Creado por Un Hilo Infinito}
	\label{ELEC_CAMPOHILOINFINITO}
\end{figure}

\begin{equation}\label{ELEC_HILOINFINITO}
	E = 2k\frac{\lambda}{r} = \frac{\lambda}{2\varepsilon\pi r}
\end{equation}

Siendo $\lambda$ las cargas por longitud $\frac{\textnormal{cargas ($C$)}}{\textnormal{longitud ($m$)}}$ y $\varepsilon$ depende del material.

\begin{mybox}{El Electronvoltio}
	Un Electronvoltio se define como la energía que tiene un electrón sometido a una diferencia de potencial de \SI{1}{\volt}.
	
	\[
		\qty{1}{\electronvolt} = \qty{1.602e-19}{\joule}
	\]
\end{mybox}

	\begin{mybox}{Conceptos Clave: Campo Eléctrico}
		\begin{itemize}
			\item \textbf{Vectores vs Escalares:}
			\begin{itemize}
				\item Fuerza ($\vec{F}$) y Campo ($\vec{E}$) son \textbf{vectores}
				\item Potencial ($V$) y Energía ($E_p$) son \textbf{escalares}
			\end{itemize}
			\item \textbf{Signo de la carga:} En las fórmulas escalares ($V, E_p$), \textbf{incluye} el signo de la carga. En las vectoriales, usa el signo para determinar el sentido del vector.
		\end{itemize}
	\end{mybox}

\clearpage

\subsection{Resumen de Fórmulas}

\begin{table}[H]
	\centering
	\renewcommand{\arraystretch}{1.5}
	\setlength{\tabcolsep}{6pt}
	\setlength{\aboverulesep}{0pt}
	\setlength{\belowrulesep}{0pt}
	\renewcommand{\tabularxcolumn}[1]{m{#1}}
	\rowcolors{2}{gray!10}{white}
	\begin{tabularx}{\textwidth}{
			>{\centering\arraybackslash}m{4.5cm}
			>{\centering\arraybackslash}X
			>{\centering\arraybackslash}m{3cm}}
		\toprule
		\rowcolor{gray!25}
		\textbf{Magnitud / Concepto} & \textbf{Fórmula} & \textbf{Unidades (SI)} \\
		\midrule
		
		\textsc{Ley de Coulomb} & $\displaystyle \vec{F} = k\frac{Qq}{r^2}\vec{u_r}$ & \unit{\newton} \\
		
		\textsc{Constante de Coulomb} & $\displaystyle k = \SI{9e9}{} = \frac{1}{4\pi\varepsilon}$ & \unit{\newton\meter\squared\per\coulomb\squared} \\
		
		\textsc{Intensidad del Campo Eléctrico (Definición)} & $\displaystyle \vec{F} = q\vec{E}$ & \unit{\newton} \\
		
		\textsc{Campo Eléctrico Creado por una Carga Puntual} & $\displaystyle \vec{E} = k\frac{Q}{r^2}\vec{u_r}$ & \unit{\newton\per\coulomb} \\
		
		\textsc{Principio de Superposición} & $\displaystyle \vec{F}_T = \sum_i \vec{F}_i$ & \unit{\newton} \\
		
		\textsc{Energía Potencial Eléctrica} & $\displaystyle E_p = k\frac{Qq}{r}$ & \unit{\joule} \\
		
		\textsc{Potencial Eléctrico} & $\displaystyle V = \frac{E_p}{q} = k\frac{Q}{r}$ & \unit{\volt} = \unit{\joule\per\coulomb} \\
		
		\textsc{Trabajo del Campo Eléctrico} & $\displaystyle W_{A\rightarrow B} = E_{pA} - E_{pB} = -q(V_B - V_A)$ & \unit{\joule} \\
		
		\textsc{Campo Eléctrico Uniforme} & $\displaystyle E = \frac{F}{q}$ & \unit{\newton\per\coulomb} \\
		
		\textsc{Campo Creado por Láminas Infinitas} & $\displaystyle E = \frac{\sigma}{\varepsilon}$ & \unit{\newton\per\coulomb} \\
		
		\textsc{Campo Creado por un Hilo Infinito} & $\displaystyle E = 2k\frac{\lambda}{r} = \frac{\lambda}{2\pi\varepsilon r}$ & \unit{\newton\per\coulomb} \\
		
		\textsc{Relación entre $V$ y $E$} & $\Delta V = -\vec{E} \cdot \Delta\vec{r}$ & --- \\
		
		\textsc{Electron-voltio} & $\displaystyle \qty{1}{\electronvolt} = \num{1.602e-19}$ & \unit{\joule} \\
		
		\bottomrule
	\end{tabularx}
	\caption{Formulario del Campo Eléctrico}
	\label{ELEC_TABLAFORMULARIO}
\end{table}