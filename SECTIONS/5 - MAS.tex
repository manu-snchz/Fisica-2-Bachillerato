\section{Movimiento Armónico Simple}

\subsection{Movimiento Armónico Simple}

\begin{enumerate}
	\item \textsc{\textbf{Movimiento Periódico}}: Es un movimiento que se \textbf{repite} cada cierto intervalo de \textbf{tiempo}, por ejemplo, el movimiento de los planetas, el péndulo de un reloj o un muelle que vibra.
	\item \textsc{\textbf{Movimiento Oscilatorio}}: Es un movimiento que tiene un cuerpo que se \textbf{desplaza} a un lado y a otro de su posición de \textbf{equilibrio}, de forma \textbf{periódica}. Por ejemplo, el péndulo de un reloj o un muelle que vibra.
	\item \textsc{\textbf{Movimiento Vibratorio}}: Es un movimiento \textbf{oscilatorio} con una \textbf{trayectoria rectilínea}. Por ejemplo, un muelle que se comprime.
\end{enumerate}

En todos los movimientos periódicos denominamos:

\begin{enumerate}
	\item \textbf{Periodo} ($T$): Intervalo de \textbf{tiempo} que tarda un cuerpo en \textbf{repetir} su movimiento. Medido en segundos (\unit{\second}).
	\item \textbf{Frecuencia} ($f$): Número de \textbf{repeticiones} que realiza un cuerpo en un \textbf{segundo}. Medida en Hercios (\unit{\hertz}) o la inversa del segundo (\unit{\per\second}).
\end{enumerate}
	
Ambas \textbf{magnitudes} están \textbf{relacionadas} por la expresión:

\begin{equation}
f = \frac{1}{T} \Longleftrightarrow T = \frac{1}{f}
\end{equation}


El movimiento armónico simple es el movimiento que tienen los cuerpos que se mueven por acción de una fuerza restauradora proporcional a la distancia que separa al cuerpo de su posición de equilibrio. Viene dada por la ecuación:

\begin{equation}\label{MAS_ECUACION}
    x(t) = A \cdot \sin(\omega \cdot t + \varphi_0)
\end{equation}

Donde:
\begin{enumerate}
    \item $x(t)$: Elongación (\unit{m})
    \item $A$: Amplitud (\unit{m})
    \item $\omega$: Frecuencia o velocidad angular (\unit{\radian\per\second})
    \item $\varphi_0$: Fase inicial o desfase (\unit{\radian})
    \item $\omega \cdot t + \varphi_0$: Fase (\unit{\radian})
\end{enumerate}

A partir de la ecuación \ref{MAS_ECUACION} podemos obtener la velocidad y la aceleración del movimiento armónico simple.

\begin{equation}\label{MAS_VELOCIDAD}
    v(t) = \frac{dx}{dt} = A \cdot \omega \cdot \cos(\omega \cdot t + \varphi_0)
\end{equation}

\begin{equation}\label{MAS_ACELERACION}
    a(t) = \frac{dv}{dt} = \frac{d^2x}{dt^2}= -A \cdot \omega^2 \cdot \sin(\omega \cdot t + \varphi_0)
\end{equation}

A partir de éstas podemos obtener la velocidad y la aceleración en función de la posición:

\begin{equation}\label{MAS_VELOCIDADPOSICION}
    v(x) = \pm \omega \cdot \sqrt{A^2 - x^2}
\end{equation}

\begin{equation}\label{MAS_ACELERACIONPOSICION}
    a(x) = - \omega^2 \cdot x
\end{equation}

\subsubsection{Representación Gráfica del MAS}

% \begin{figure}[h!]
%     \centering
%     \begin{tikzpicture}[scale=1.1]
%         % Parámetros
%         \def\A{1}
%         \def\T{4}
%         \def\w{2*pi/\T}

%         % Ejes
%         \draw[->, thick] (0,-1) -- (0,1) node[above] {};
%         \draw[->, thick] (-0.2,0) -- (10,0) node[right] {$t$};

%         % Marca de amplitud A
%         \draw[thick] (-0.15,\A) -- (0.15,\A);
%         \node[left] at (0,\A) {$A$};

%         % Onda MAS
%         \draw[very thick, red]
%         plot[domain=0:10, samples=200]
%         (\x,{\A*sin(\w*\x r)});

%         % Marcas de T y 2T
%         \draw[thick] (\T,0.15) -- (\T,-0.15);
%         \node[below] at (\T,-0.2) {$T$};

%         \draw[thick] (2*\T,0.15) -- (2*\T,-0.15);
%         \node[below] at (2*\T,-0.2) {$2T$};

%         % Línea vertical discontinua en T
%         \draw[dashed] (\T,-1) -- (\T,1);

%         % Flecha periodo
%         \draw[<->, thick] (0,-1) -- (\T,-1);
%         \node[below] at (\T/2,-1) {$T = \dfrac{2\pi}{\omega}$};
%     \end{tikzpicture}
%     \caption{Representación gráfica del MAS de la amplitud con respecto del tiempo}
%     \label{MAS_GRFICAMAS}
% \end{figure}

\begin{figure}[H]
	\centering
	\includegraphics[scale=0.5]{MAS_MUELLE}
	\caption{Muelle oscilante simple}
	\label{MAS_MUELLE}
\end{figure}

\begin{figure}[H]
	\centering
	\includegraphics[scale=0.5]{MAS_PENDULO}
	\caption{Péndulo simple}
	\label{MAS_PENDULO}
\end{figure}

\clearpage

\subsection{Dinámica del movimiento armónico simple}

\subsubsection{Dinámica y periodo de un muelle}

Ley de Hooke:

\begin{equation}\label{MAS_HOOKE}
    F = -k \cdot x
\end{equation}

Siendo $k$ la constante elástica del muelle, que se calcula con:

\begin{equation}\label{MAS_K}
    F = m\cdot a \Longrightarrow k = m \cdot \omega^2
\end{equation}

Despejando la fórmula \ref{MAS_K} obtenemos:

\begin{equation}\label{MAS_PERIODODINAMICA}
    T = 2\pi \cdot \sqrt{\frac{m}{k}}
\end{equation}

\subsubsection{Dinámica y periodo de un péndulo simple}

En un péndulo simple, una masa $m$ atada a un hilo oscila respecto a un punto de equilibrio bajo el efecto de un campo gravitatorio $g$. El periodo se puede calcular mediante:

\begin{equation}\label{MAS_PERIODODEPENDULO}
    T = 2\pi \cdot \sqrt{\frac{L}{g}}
\end{equation}

\subsection{Energía en el movimiento armónico simple}

Una partícula en un MAS tendrá energía potencial y cinética:

\begin{equation}\label{MAS_ENERGIAPOTENCIAL}
    E_p = \frac{1}{2} \cdot k \cdot x^2
\end{equation}

\begin{equation}\label{MAS_ENERGIACINETICA}
    E_c = \frac{1}{2} \cdot m \cdot v^2
\end{equation}

Si la partícula se encuentra en $x = A$, la velocidad será 0, por lo que su energía potencial durante todo su movimiento será:

\begin{equation}\label{MAS_ENERGIAPOTENCIALMAXIMA}
    E_p = \frac{1}{2} \cdot k \cdot A^2
\end{equation}
    