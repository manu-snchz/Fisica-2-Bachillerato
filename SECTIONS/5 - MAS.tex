\section{Movimiento Armónico Simple}

\subsection{Movimiento Armónico Simple}

\begin{enumerate}
	\item \textsc{\textbf{Movimiento Periódico}}: Es un movimiento que se \textbf{repite} cada cierto intervalo de \textbf{tiempo}, por ejemplo, el movimiento de los planetas, el péndulo de un reloj o un muelle que vibra.
	\item \textsc{\textbf{Movimiento Oscilatorio}}: Es un movimiento que tiene un cuerpo que se \textbf{desplaza} a un lado y a otro de su posición de \textbf{equilibrio}, de forma \textbf{periódica}. Por ejemplo, el péndulo de un reloj o un muelle que vibra.
	\item \textsc{\textbf{Movimiento Vibratorio}}: Es un movimiento \textbf{oscilatorio} con una \textbf{trayectoria rectilínea}. Por ejemplo, un muelle que se comprime.
\end{enumerate}

En todos los movimientos periódicos denominamos:

\begin{enumerate}
	\item \textbf{Periodo} ($T$): Intervalo de \textbf{tiempo} que tarda un cuerpo en \textbf{repetir} su movimiento. Medido en segundos (\unit{\second}).
	\item \textbf{Frecuencia} ($f$): Número de \textbf{repeticiones} que realiza un cuerpo en un \textbf{segundo}. Medida en Hercios (\unit{\hertz}) o la inversa del segundo (\unit{\per\second}).
\end{enumerate}
	
Ambas \textbf{magnitudes} están \textbf{relacionadas} por la expresión:

\begin{equation}
f = \frac{1}{T} \Longleftrightarrow T = \frac{1}{f}
\end{equation}


El movimiento armónico simple es el movimiento que tienen los cuerpos que se mueven por acción de una fuerza restauradora proporcional a la distancia que separa al cuerpo de su posición de equilibrio. Viene dada por la ecuación:

\begin{equation}\label{MAS_ECUACION}
    x(t) = A \cdot \sin(\omega \cdot t + \varphi_0)
\end{equation}

Donde:
\begin{enumerate}
    \item $x(t)$: Elongación (\unit{m})
    \item $A$: Amplitud (\unit{m})
    \item $\omega$: Velocidad angular (\unit{\radian\per\second})
    \item $\varphi_0$: Fase inicial o desfase (\unit{\radian})
    \item $\omega \cdot t + \varphi_0$: Fase (\unit{\radian})
\end{enumerate}

A partir de la ecuación \ref{MAS_ECUACION} podemos obtener la velocidad y la aceleración del movimiento armónico simple.

\begin{equation}\label{MAS_VELOCIDAD}
    v(t) = \frac{dx}{dt} = A \cdot \omega \cdot \cos(\omega \cdot t + \varphi_0)
\end{equation}

\begin{equation}\label{MAS_ACELERACION}
    a(t) = \frac{dv}{dt} = \frac{d^2x}{dt^2}= -A \cdot \omega^2 \cdot \sin(\omega \cdot t + \varphi_0)
\end{equation}

A partir de éstas podemos obtener la velocidad y la aceleración en función de la posición:

\begin{equation}\label{MAS_VELOCIDADPOSICION}
    v(x) = \pm \omega \cdot \sqrt{A^2 - x^2}
\end{equation}

\begin{equation}\label{MAS_ACELERACIONPOSICION}
    a(x) = - \omega^2 \cdot x
\end{equation}

La ecuación \ref{MAS_VELOCIDADPOSICION} puede deducirse a partir de \ref{MAS_ECUACION} y \ref{MAS_VELOCIDAD}, elevando ambas al cuadrado y utilizando la identidad trigonométrica
$\sin^2\varphi + \cos^2\varphi = 1$, se obtiene:

\begin{equation}
v^2 = \omega^2 (A^2 - x^2)
\end{equation}

\subsection{Dinámica del Movimiento Armónico Simple}

\subsubsection{Dinámica y Periodo de un Muelle}

\begin{wrapfigure}{L}{0.4\textwidth}
	\centering
	\includegraphics[scale=0.43]{MAS_MUELLE}
	\caption{Muelle oscilante simple}
	\label{MAS_MUELLE}
\end{wrapfigure}


Ley de Hooke:

\begin{equation}\label{MAS_HOOKE}
    F = -k \cdot x
\end{equation}

Siendo $k$ la constante elástica del muelle, que se calcula con:

\begin{equation}\label{MAS_K}
    F = m\cdot a \Longrightarrow k = m \cdot \omega^2
\end{equation}

Despejando la fórmula \ref{MAS_K} obtenemos:

\begin{equation}\label{MAS_PERIODODINAMICA}
    T = 2\pi \cdot \sqrt{\frac{m}{k}}
\end{equation}

\subsubsection{Dinámica y Periodo de un Péndulo Simple}

\begin{figure}[H]
	\centering
	\includegraphics[scale=0.6]{MAS_PENDULO}
	\caption{Péndulo simple}
	\label{MAS_PENDULO}
\end{figure}

En un péndulo simple, una masa $m$ atada a un hilo oscila respecto a un punto de equilibrio bajo el efecto de un campo gravitatorio $g$. El periodo se puede calcular mediante:

\begin{equation}\label{MAS_PERIODODEPENDULO}
    T = 2\pi \cdot \sqrt{\frac{L}{g}}
\end{equation}

\subsection{Energías en el Movimiento Armónico Simple}

Una partícula en un MAS tendrá energía tanto potencial como cinética:

\begin{figure}[H]
    \centering
    \includegraphics[scale=0.3]{MAS_GRAFICA_ENERGIAS}
    \caption{Graficas de las energías en el MAS}
    \label{MAS_GRAFICA_ENERGIAS}
\end{figure}

\subsubsection{Energías en el MAS según la posición}

\begin{equation}\label{MAS_ENERGIAPOTENCIALPOSICION}
    E_p = \frac{1}{2} \cdot k \cdot x^2
\end{equation}

\begin{equation}\label{MAS_ENERGIACINETICAPOSICION}
    E_c = \frac{1}{2} \cdot m \cdot \omega^2 \cdot (A^2 - x^2)
\end{equation}

\subsubsection{Energías en el MAS según el tiempo}

\begin{equation}\label{MAS_ENERGIAPOTENCIALTIEMPO}
    E_p = \frac{1}{2} \cdot k \cdot A^2 \cdot \sin^2(\omega t + \varphi_0)
\end{equation}

\begin{equation}\label{MAS_ENERGIACINETICATIEMPO}
    E_c = \frac{1}{2} \cdot m \cdot \omega^2 \cdot A^2 \cdot \cos^2(\omega t + \varphi_0)
\end{equation}

Si la partícula se encuentra en $x = A$, la velocidad será 0, por lo que su energía potencial durante todo su movimiento será: $E_m = \frac{1}{2} \cdot k \cdot A^2$. Como la energía mecánica no varía en todo el movimiento, la energía mecánica en todo momento del movimiento será:

\begin{equation}\label{MAS_ENERGIAMECANICAMAXIMA}
    E_m = \frac{1}{2} \cdot k \cdot A^2
\end{equation}

\begin{mybox}{Conceptos Clave: Movimiento Armónico Simple}
	\begin{itemize}
		\item \textbf{Estados límite:} En $x=\pm A$ la velocidad es nula y la aceleración máxima; en $x=0$ la aceleración es nula y la velocidad máxima. Esto permite razonar el movimiento sin cálculos.
		
		\item \textbf{Uso de la fase:} La fase inicial $\varphi_0$ se determina a partir de las condiciones iniciales $(x_0, v_0)$ y fija completamente el movimiento.
		
		\item \textbf{Evitar errores típicos:} El signo de $v$ indica el \textbf{sentido del movimiento}, no su módulo; el signo de $x$ determina siempre el signo de la aceleración.
	\end{itemize}
\end{mybox}

\clearpage

\subsection{Resumen de Fórmulas}

\begin{table}[H]
	\centering
	\renewcommand{\arraystretch}{1.5}
	\setlength{\tabcolsep}{6pt}
	\setlength{\aboverulesep}{0pt}
	\setlength{\belowrulesep}{0pt}
	\renewcommand{\tabularxcolumn}[1]{m{#1}}
	\rowcolors{2}{gray!10}{white}
	\begin{tabularx}{\textwidth}{
			>{\centering\arraybackslash}m{5cm}
			>{\centering\arraybackslash}X
			>{\centering\arraybackslash}m{3cm}}
		\toprule
		\rowcolor{gray!25}
		\textbf{Magnitud / Concepto} & \textbf{Expresión} & \textbf{Unidades (SI)} \\
		\midrule
	
		\textsc{Ecuación del MAS} & $\displaystyle x(t) = A \cdot \sin(\omega t + \varphi_0)$ & \unit{\meter} \\

		\textsc{Velocidad Angular} & $\displaystyle \omega = \frac{2\pi}{T}$ & \unit{\radian\per\second} \\

		\textsc{Velocidad} & $\displaystyle v(t) = A \cdot \omega \cdot \cos(\omega t + \varphi_0)$ & \unit{\meter\per\second} \\

		\textsc{Aceleración} & $\displaystyle a(t) = -A \cdot \omega^2 \cdot \sin(\omega t + \varphi_0)$ & \unit{\meter\per\second\squared} \\

		\textsc{Velocidad en función de $x$} & $\displaystyle v(x) = \pm \omega \sqrt{A^2 - x^2}$ & \unit{\meter\per\second} \\

		\textsc{Aceleración en función de $x$} & $\displaystyle a(x) = -\omega^2 x$ & \unit{\meter\per\second\squared} \\

		\textsc{Ley de Hooke} & $\displaystyle F = -k \cdot x$ & \unit{\newton} \\

		\textsc{Período de un Muelle} & $\displaystyle T = 2\pi \sqrt{\frac{m}{k}}$ & \unit{\second} \\

		\textsc{Período del péndulo simple} & $\displaystyle T = 2\pi \sqrt{\frac{L}{g}}$ & \unit{\second} \\

		\textsc{Energía Potencial en el MAS} & $\displaystyle E_p = \frac{1}{2} k x^2$ & \unit{\joule} \\

		\textsc{Energía Cinética en el MAS} & $\displaystyle E_c = \frac{1}{2} m v^2$ & \unit{\joule} \\

		\textsc{Energía Mecánica en el MAS} & $\displaystyle E_m = \frac{1}{2} k A^2$ & \unit{\joule} \\

		\bottomrule
	\end{tabularx}
	\caption{Formulario del Movimiento Armónico Simple}
	\label{MAS_TABLAFORMULARIO}
\end{table}