\section{Inducción Electromagnética}

\subsection{El Flujo Magnético}

El \textbf{flujo magnético} ($\phi$) se define, en una \textbf{superficie}, como el número de \textbf{líneas de inducción} que la \textbf{atraviesan}. Este número número de \textbf{líneas por superficie} perpedincular a las mismas indica la intensidad del campo, es decir:

\begin{equation}\label{INDUCC_FLUJO1}
	\phi = \vec{B} \cdot \vec{S} \Longrightarrow \phi = B \cdot S \cdot \cos \varphi 
\end{equation}

El flujo magnético se mide en \textbf{Wéber} (\unit{\weber}). Además, cuando se trata de una \textbf{bobina}, el flujo magnético $\varphi$ es $N$ veces el campo magnético de una espira, siendo $N$ el número de <<\textit{vueltas}>> que tiene la bobina.

\begin{figure}[H]
	\centering
	\includegraphics[scale=0.17]{INDUCC_SITUACIONESFLUJO}
	\caption{Situaciones del Flujo Magnético}
	\label{INDUCC_SITUACIONESFLUJO}
\end{figure}

\subsection{Ley de Faraday-Lenz}

La Ley de Faraday-Lenz establece que cuando el \textbf{flujo magnético} que atraviesa un circuito varía, aparece una \textbf{fuerza electromotriz} (\textit{fem}) inducida. Su valor viene dado por:

\begin{equation}\label{INDUCC_LEYFARADAYLENZ}
\varepsilon = -\frac{d\phi}{dt}
\end{equation}

El signo \textbf{negativo} de la fórmula \ref{INDUCC_LEYFARADAYLENZ} indica que la \textbf{corriente} inducida \textbf{siempre se opone} al cambio de flujo que la produce (sentido de Lenz). La \textbf{fem} \(\varepsilon\) es la energía por unidad de carga que surge en el circuito debido a esa variación de flujo y se mide en voltios (\unit{\volt}). \textbf{Ejemplo:} Si un imán se \textbf{acerca} a una espira, el flujo magnético aumenta y aparece una corriente cuyo campo se \textbf{opone} a ese acercamiento. Si el imán se \textbf{aleja}, el flujo \textbf{disminuye} y la corriente inducida \textbf{cambia} de \textbf{sentido} para intentar mantener el flujo \textbf{anterior}.

\clearpage

\subsection{La Experiencia de Henry}

Los estudios de Joseph Henry acerca del electromagnetismo fueron de gran importancia para conocer las leyes que rigen este campo de la física. Supongamos un \textbf{hilo} conductor de \textbf{longitud} $\ell$ que se mueve hacia la \textbf{derecha} con \textbf{velocidad constante} $\vec{v}$ en el seno de un campo \textbf{magnético} perpendicular entrante en el plano del hilo, tal y como se indica en la figura \ref{INDUCC_EXPERIENCIAHENRY}.

\begin{figure}[H]
	\centering
	\includegraphics[scale=0.3]{INDUCC_EXPERIENCIAHENRY}
	\caption{}
	\label{INDUCC_EXPERIENCIAHENRY}
\end{figure}

La $\vec{F_B}$ hace que los \textbf{electrones} se desplacen hacia la parte \textbf{inferior} del conductor($b$ en la figura \ref{INDUCC_EXPERIENCIAHENRY}). Gracias a eso, aparecerá un campo eléctrico que ejerce una \textbf{fuerza} sobre los electrones en \textbf{sentido opuesto} a $\vec{F_B}$. Se produce una situación de \textbf{equilibrio} y ya no hay más separación de cargas.

\begin{equation}
	\tag{\ref{MAGN_LORENTZ}}
	\vec{F_B} = \vec{v} \times \vec{B}
\end{equation}

\begin{equation}
	\tag{\ref{ELEC_FCAMPOELECTRICO}}
	\vec{F_E} = q \cdot \vec{E}
\end{equation}

\begin{equation}\label{INFUCC_EXPERIENCIAHENRY1}
	\vec{F_B} = \vec{F_E} \Longrightarrow \cancel{q} \vec{v} \times \vec{B} = \cancel{q} \cdot \vec{E} \Longrightarrow \vec{E} = \vec{v} \times \vec{B}
\end{equation}

\begin{equation}\label{INFUCC_EXPERIENCIAHENRY2}
	\Delta V = E \cdot \ell = \vec{v} \times \vec{B} \cdot \ell
\end{equation}


\clearpage

\begin{mybox}{Ejercicio Ejemplo}
	\textbf{Una espira triangular de $\SI{4}{\meter}$ de lado se desplaza a $\SI{2}{\meter\per\second}$ hacia una región donde hay un campo magnético $\vec{B}$ perpendicular al plano de la espira, tal y como se indica en la figura \ref{INDUCC_EJEMPLO1}. En $t = \SI{0}{\second}$ la espira está a $\SI{2}{\meter}$ de la región.}

	\begin{figure}[H]
		\centering
		\includegraphics[scale=0.1]{INDUCC_EJEMPLO1}
		\caption{}
		\label{INDUCC_EJEMPLO1}
	\end{figure}
	
	\textbf{Indica la expresión de la fem inducida en la espira cuando penetra en la región del campo magnético. Calcula el valor del campo si en el instante $t = \SI{2}{\second}$ la fem inducida es $\varepsilon = \SI{1,6}{\volt}$.}

	---

	Podemos calcular primero el flujo magnético que atraviesa la espira:

	\begin{equation}
		\phi = B \cdot S \cdot \cancelto{1}{\cos \varphi}
	\end{equation}

	Donde $S$ es el área de la espira y $\varphi$ es el ángulo, en este caso $\varphi = \SI{0}{\degree}$, entre el campo magnético y el plano de la espira. Se puede ver que la superficie es un triángulo rectángulo, por lo que su área es $S = \frac{\ell^2}{2}$. Asimismo, al tratarse de un MRU, podemos sustituir $\ell$ por $v \cdot t$, lo que nos queda:

	\begin{equation}
		\phi = B \cdot \frac{v^2 t^2}{2}
	\end{equation}

	Sustituyendo los datos que ya tenemos y derivando para obtener la fem inducida:

	\begin{equation}
		\phi (t) = B \frac{4t^2}{2} = 2Bt^2  \Longrightarrow \varepsilon = -\frac{d\phi}{dt} = -4Bt
	\end{equation}

	Sin embargo, en esta ecuación no se está teniendo en cuenta que la espira se encuentra a 2 metros de la región del campo magnético, por lo que el tiempo que usaremos será $t_{\textnormal{desfase}} = \frac{\SI{2}{\meter}}{\SI{2}{\meter\per\second}} = \SI{1}{\second}$. El tiempo que usaremos en la ecueción serán los dos segundos del enunciado menos el segundo de desfase $t = \SI{2}{\second} - \SI{1}{\second} = \SI{1}{\second}$. En el enunciado también nos decían que $\varepsilon = \SI{1,6}{\volt}$, por lo que:

	\begin{equation}
		\SI{1.6}{\volt} = -4B \cdot \SI{1}{\second} \Longrightarrow B = \frac{\SI{1.6}{\volt}}{4 \cdot \SI{1}{\second}} = \SI{0.4}{\tesla}
	\end{equation}

\end{mybox}

\subsection{Aplicaciones de la inducción electromagnética}

\subsubsection{Generadores eléctricos}

Un \textbf{generador eléctrico} es un dispositivo \textbf{capaz} de transformar algún tipo de energía en energía \textbf{eléctrica}. Nikola Tesla inventó el generador de \textbf{corriente alterna} o \textbf{alternador}. Modificándolo, se creó el generador de corriente continua llamado \textbf{dinamo}.

\paragraph{Alternador}

Un alternador es un \textbf{generador} de corriente \textbf{alterna}. El sentido en el que circulan las cargas \textbf{cambian periodicamente}. La Ley de Faraday (fórmula \ref{INDUCC_LEYFARADAYLENZ}) nos permite obtener la función de la \textbf{fem} en el alternador con respecto al \textbf{tiempo}:

\begin{equation}
	\varepsilon = -\frac{d\phi}{dt} = -\frac{d (B\cdot S\cdot \cos (\omega t))}{dt} = \omega \cdot B \cdot S \cdot \sin (\omega t)
\end{equation}

La \textbf{fem} máxima es $\varepsilon_{\textnormal{máxima}} = \omega \cdot B \cdot S$. Donde $\omega$ es la \textbf{velocidad angular}, con la que se puede calcular la \textbf{frecuencia}: $f = \frac{\omega}{2\pi}$.

\paragraph{Dinamo}

Una dinamo es un \textbf{generador} de corriente \textbf{continua}, denominada así porque \textbf{no cambia} el \textbf{sentido} en que circulan las cargas eléctricas. Se puede obtener \textbf{modificando} el diseño del \textbf{alternador}, utilizando un único anillo partido en dos y conectando a cada una de las mitades una \textbf{escobilla}.

\begin{figure}[H]
	\centering
	\begin{subfigure}{0.45\textwidth}
		\centering
		\includegraphics[scale=0.6]{INDUCC_ALTERNADOR}
		\caption{Alternador}
		\label{INDUCC_ALTERNADOR}
	\end{subfigure}
	\begin{subfigure}{0.45\textwidth}
		\centering
		\includegraphics[scale=0.6]{INDUCC_DINAMO}
		\caption{Dinamo}
		\label{INDUCC_DINAMO}
	\end{subfigure}
	\caption{Gráficas de fem inducida en los distintos tipos de generadores}
	\label{INDUCC_GENERADORES}
\end{figure}

Tanto en \ref{INDUCC_ALTERNADOR} como en \ref{INDUCC_DINAMO}, el generador ha dado una \textbf{vuelta entera} ($2\pi \unit{\radian}$), es decir, ha pasado \textbf{dos veces} por \SI{0}{\volt}.

\paragraph{Transformador}

Un transformador es un dispositivo empleado para \textbf{modificar} el \textbf{voltaje} o la \textbf{intensidad} de una \textbf{corriente alterna}. Éste consta de dos \textbf{bobinas} enrolladas en torno a un \textbf{núcleo} de hierro, aisladas entre sí. Por una de las bobinas se hace pasar una corriente de \textbf{entrada} (llamada \textbf{primaria}) con un numero $N_p$ ($N_2$ en la figura \ref{INDUCC_TRANS}) de espiras. En la otra (llamada \textbf{secundaria}) obtendremos la corriente de \textbf{salida} con un numero $N_s$ ($N_1$ en la figura \ref{INDUCC_TRANS}) de espiras.

De las ecuaciones de la fem y del flujo, podemos hallar la ecuación del transformador:

\begin{equation}\label{INDUCC_TRANSFORMADOR}
	\frac{V_s}{V_p} = \frac{N_s}{N_p} = \frac{I_p}{I_s}
\end{equation}

\begin{figure}[H]
	\centering
	\includegraphics[scale=0.3]{INDUCC_TRANS}
	\caption{Esquema de un transformador}
	\label{INDUCC_TRANS}
\end{figure}

\begin{mybox}{Conceptos Clave: Inducción Electromagnética}
	\begin{itemize}
		\item La fuerza electromotriz (fem, \(\varepsilon\)) \textbf{no} es una \textbf{fuerza}, sino la \textbf{energía} por unidad de carga. En el SI, la fem se mide en \textbf{voltios} (\unit{\volt}).
		\item Un flujo magnético muy intenso \textbf{no} garantiza una \textit{fem} inducida. Solo existe inducción si hay una \textbf{variación temporal} del flujo ($ \frac{d\phi}{dt} \neq 0$). Si el flujo es constante, la fem es nula.
		\item El vector superficie \(\vec{S}\) es \textbf{perpendicular} al plano de la espira. Por tanto, si el \textbf{campo} fuera \textbf{paralelo} al plano de la espira, el \textbf{ángulo} con el vector superficie es \(\SI{90}{\degree}\) y por tanto el flujo magnético es \(\phi = \SI{0}{\weber}\).
		\item Recuerda la \textbf{Ley de Ohm} para obtener con la \textit{fem} la \textbf{corriente} inducida: \(\varepsilon = I \cdot R\).
	\end{itemize}
\end{mybox}


\clearpage

\subsection{Resumen de Fórmulas}

\begin{table}[H]
	\centering
	\renewcommand{\arraystretch}{1.5}
	\setlength{\tabcolsep}{6pt}
	\setlength{\aboverulesep}{0pt}
	\setlength{\belowrulesep}{0pt}
	\renewcommand{\tabularxcolumn}[1]{m{#1}}
	\rowcolors{2}{gray!10}{white}
	\begin{tabularx}{\textwidth}{
			>{\centering\arraybackslash}m{4.5cm}
			>{\centering\arraybackslash}X
			>{\centering\arraybackslash}m{3cm}}
		\toprule
		\rowcolor{gray!25}
		\textbf{Magnitud / Concepto} & \textbf{Fórmula} & \textbf{Unidades (SI)} \\
		\midrule
		
		\textsc{Flujo Magnético} & $\displaystyle \phi = \vec{B} \cdot \vec{S} \Longrightarrow \phi = B \cdot S \cdot \cos \varphi$ & \unit{\weber} \\
		
		\textsc{Ley de Faraday--Lenz} & $\displaystyle \varepsilon = -\frac{d\phi}{dt}$ & \unit{\volt} \\

		\textsc{Experiencia de Henry} & $\displaystyle \Delta V = v \cdot B \cdot \ell$ & \unit{\volt} \\

		\textsc{Ley de Ohm} & $\displaystyle I = \frac{\varepsilon}{R}$ & \unit{\ampere} \\
		
		\bottomrule
	\end{tabularx}
	\caption{Formulario de Inducción Electromagnética}
	\label{INDUCC_TABLAFORMULARIO}
\end{table}