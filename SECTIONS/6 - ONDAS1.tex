\section{Ondas. El Sonido}

\textbf{Una onda} es una perturbación que se propaga a través de un medio o del espacio, produciendo un \textbf{transporte de energía} sin que exista un \textbf{transporte neto de materia}. El movimiento ondulatorio se describe, por tanto, como la propagación de una perturbación que transmite energía sin desplazar las partículas del medio de forma permanente. Un ejemplo claro es una ola en el mar: las moléculas de agua no avanzan con la ola, sino que \textbf{oscilan alrededor de su posición de equilibrio}, transmitiendo la perturbación a las moléculas vecinas.

\subsection{Tipos de ondas}

\subsubsection{Según las dimensiones de propagación}

\begin{enumerate}
    \item \textsc{\textbf{Unidimensionales}}: Se propagan en una dirección. Como por ejemplo, la onda en una cuerda.
    \item \textsc{\textbf{Bidimensionales}}: Se propagan en dos direcciones. Como por ejemplo, la onda en una superficie.
    \item \textsc{\textbf{Tridimensionales}}: Se propagan en tres direcciones. Como por ejemplo, la onda en el aire.
\end{enumerate}

\subsubsection{Según el medio de propagación}

\begin{enumerate}
    \item \textsc{\textbf{Ondas Mecánicas}}: Necesitan un medio material para propagarse. Por ejemplo, el sonido.
    \item \textsc{\textbf{Ondas Electromagnéticas}}: No necesitan un medio material para propagarse. Por ejemplo, la luz.
\end{enumerate}

\subsubsection{Según la dirección en que vibran las partículas con relación a la dirección de avance}

\begin{enumerate}
    \item \textsc{\textbf{Ondas Longitudinales}}: Las partículas vibran en la misma dirección que la dirección de avance de la onda. Como por ejemplo, el sonido.
    \item \textsc{\textbf{Ondas Transversales}}: Las partículas vibran perpendicular a la dirección de avance de la onda. Como por ejemplo, las vibraciones en el agua.
\end{enumerate}


\subsection{Ecuación Matemática de la Onda Armónica}

\begin{equation}\label{ONDAS1_ECUACION_ARMÓNICA}
    y(x,t) = A \cdot \sin(\omega \cdot t \pm k \cdot x + \varphi_0)
\end{equation}

\subsection{Magnitudes de la Onda Armónica}

\begin{enumerate}
    \item $y$: \textbf{Elongación de la onda}. Es el desplazamiento instantáneo de una partícula del medio respecto a su posición de equilibrio en un punto y en un instante determinados. Se mide en metros ($\unit{\meter}$).

    \item $A$: \textbf{Amplitud de la onda}. Es el valor máximo del módulo de la elongación que alcanza una partícula del medio durante el movimiento ondulatorio. Se mide en metros ($\unit{\meter}$).

    \item $\omega$: \textbf{Frecuencia o velocidad angular}. Es la velocidad de variación de la fase de la onda con el tiempo. Se mide en radianes por segundo ($\unit{\radian\per\second}$). Se define como:
    \[
    \omega = 2\pi f
    \]

    \item $k$: \textbf{Número de onda}. Es la velocidad de variación de la fase de la onda con la posición espacial. Se mide en radianes por metro ($\unit{\radian\per\meter}$). Se define como:
    \[
    k = \frac{2\pi}{\lambda}
    \]

    \item $\varphi_0$: \textbf{Fase inicial o constante de fase}. Es el valor de la fase de la onda en el origen de coordenadas espaciales y temporales. Se mide en radianes ($\unit{\radian}$).

    \item $\varphi$: \textbf{Fase de la onda}. Es la magnitud adimensional que determina el estado de vibración de un punto del medio en un instante dado. Se mide en radianes ($\unit{\radian}$).

    \item $V_p$: \textbf{Velocidad de propagación de la onda}. Es la velocidad a la que se propaga una superficie de fase constante (por ejemplo, una cresta) a través del medio. Se mide en metros por segundo ($\unit{\meter\per\second}$). Se expresa como:
    \[
    V_p = \lambda f = \frac{\omega}{k}
    \]

    \item $\lambda$: \textbf{Longitud de onda}. Es la distancia mínima entre dos puntos consecutivos del medio que oscilan en el mismo estado de vibración, es decir, con igual fase. Se mide en metros ($\unit{\meter}$).

    \item $f$: \textbf{Frecuencia}. Es el número de oscilaciones completas realizadas por una partícula del medio por unidad de tiempo. Se mide en hercios ($\unit{\hertz}$).

    \item $T$: \textbf{Periodo}. Es el tiempo necesario para que una partícula del medio complete una oscilación completa. Se mide en segundos ($\unit{\second}$) y se cumple que:
    \[
    T = \frac{1}{f}
    \]
\end{enumerate}

Si el signo $\pm$ de la fórmula \ref{ONDAS1_ECUACION_ARMÓNICA} es \textbf{positivo} ($+$), la onda se propaga hacia la \textbf{izquierda}. Si el signo $\pm$ es \textbf{negativo} ($-$), la onda se propaga hacia la \textbf{derecha}.

\subsubsection{Representación de la Onda Armónica}

\begin{figure}[H]
    \centering
    \includegraphics[scale=0.3]{ONDAS1_GRAFICA1}
    \caption{Grafica de la ecuación matemática de la onda armónica}
    \label{ONDAS1_GRAFICA1}
\end{figure}


\begin{mybox}{Conceptos Clave. Ondas}
    \begin{itemize}
        \item \textbf{¡Ojo!} No debe confundirse el \textbf{número de onda} $k$ con la \textbf{constante elástica de un muelle} ($k$). El número de onda está relacionado con la \textbf{longitud de onda} y mide la \textbf{variación de fase} de la onda por \textbf{unidad de longitud}, mientras que la constante elástica de un muelle cuantifica la \textbf{fuerza recuperadora} que ejerce el muelle cuando se deforma.
    \end{itemize}
\end{mybox}
