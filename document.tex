% preámbulo

\documentclass[titlepage,12pt]{article} %% tamaño del texto
\usepackage[spanish]{babel} %% idioma
\usepackage[a4paper]{geometry} %% márgenes
\usepackage[utf8]{inputenc} %% soporte para UTF-8
\usepackage[T1]{fontenc} %% mejora la codificación de salida
\renewcommand{\theenumi}{\roman{enumi}} %% números romanos para las enumeraciones (comentar si no es necesario)
\usepackage{graphicx} %% imágenes y gráficos
\graphicspath{ {FIGURES/} }
\usepackage{siunitx}
\sisetup{output-decimal-marker = {,}}
\usepackage{microtype}
\hyphenpenalty=1000 %% evitar separar las palabras con guión
\interlinepenalty=1000000 %% evitar saltos de línea entre párrafos
\usepackage[hidelinks]{hyperref} %% pone los hipervínculos en figuras y tablas de contenidos

\usepackage{bm} % usar la negrita en los símbolos matemáticos
\usepackage{color,soul}
\setlength{\parindent}{20pt} % sangrado
\usepackage{wrapfig}
\usepackage{caption}
\usepackage{subcaption}
\usepackage{pgfplots}

\usepackage[]{titlesec} %cambiar espacios en secciones
\titlespacing*{\section}{0pt}{1.2ex}{1.2ex}
\titlespacing*{\subsection}{0pt}{1ex}{1ex}
\titlespacing*{\subsubsection}{0pt}{0.5ex}{0.5ex}
\usepackage{booktabs}   % Líneas bonitas
\usepackage[table]{xcolor}  % Colores para la tabla
\usepackage{tabularx}   % Ajuste automático al ancho
\usepackage{float}
\addto\captionsspanish{\renewcommand{\tablename}{Tabla}}
\usepackage{colortbl}
\usepackage{lipsum}
\usepackage[most]{tcolorbox}
\tcbuselibrary{skins,breakable}
\newtcolorbox{mybox}[2][]{breakable,sharp corners, skin=enhancedmiddle jigsaw,parbox=false,
	boxrule=0mm,leftrule=2mm,boxsep=0mm,arc=0mm,outer arc=0mm,attach title to upper,
	after title={.\ }, coltitle=black,colback=gray!10,colframe=black, title={#2},
	fonttitle=\bfseries,#1}

\titleformat*{\section}{\Huge\bfseries\scshape}
\titleformat*{\subsection}{\Large\bfseries\scshape}
\titleformat*{\subsubsection}{\large\bfseries}
\titleformat*{\paragraph}{\large\bfseries}
\titleformat*{\subparagraph}{\large\bfseries}
%\setcounter{tocdepth}{2}
\usepackage{array}


% comentar si no se quieren puntos en la tabla de contenidos
\usepackage[makeroom]{cancel}
\usepackage{tocloft} %% forzar los puntos
\renewcommand{\cftdot}{·} %% definir el carácter de relleno (puede ser otro)
\renewcommand{\cftsecleader}{\cftdotfill{\cftdotsep}} %% aplicarlo a las secciones
\renewcommand{\cftsubsecleader}{\cftdotfill{\cftdotsep}} %% aplicarlo a las sub-secciones
\usepackage{mathtools}
\usepackage{multicol}
% título, autor, fecha
\usepackage{cleveref}

\title{\textsc{\huge{\textbf{Física 2\textsuperscript{o} Bachillerato}}}}
\author{Manuel Sánchez Abián}
\date{2025-2026}

% documento

\pgfplotsset{compat=1.18}


\begin{document}

\maketitle
\thispagestyle{empty}
\setcounter{page}{0}
\thispagestyle{empty}
\setcounter{page}{0}
\tableofcontents
\addtocontents{toc}{\protect\thispagestyle{empty}}
\thispagestyle{empty}
\clearpage
\setcounter{page}{1}


% -------------- CAMPO GRAVITATORIO --------------


\section{Campo Gravitatorio}

\subsection{Leyes de Kepler}

Las leyes de Kepler son 3 leyes acerca de las \textbf{órbitas} de los \textbf{planetas} alrededor del Sol, que se deducen a partir de la ley de gravitación universal.

\subsubsection{Primera Ley de Kepler}

Los planetas describen \textbf{órbitas elípticas} alrededor del Sol. El sol está situado en uno de los \textbf{focos} de la elipse.

\begin{figure}[H]
	\centering
	\includegraphics[scale=0.2]{GRAV_1LEYKEPLER.jpg}
	\caption{Primera Ley de Kepler}
	\label{GRAV_1LEYKEPLER}
\end{figure}

\subsubsection{Segunda Ley de Kepler}

Los planetas giran con una \textbf{velocidad areolar constante}, es decir, el vector posición (radiovector) barre áreas iguales en tiempos iguales:

\[
\frac{dA}{dt} = \textnormal{cte.}
\]

Esto quiere decir que la velocidad en el \textbf{perihelio} (punto más cercano al Sol) es \textbf{mayor} que la velocidad en el \textbf{afelio} (punto más lejano al Sol).

\begin{figure}[H]
	\centering
	\includegraphics[scale=0.3]{GRAV_2LEYKEPLER.jpeg}
	\caption{Segunda Ley de Kepler}
	\label{GRAV_2KEPLER}
\end{figure}

\subsubsection{Tercera Ley de Kepler}

El \textbf{cuadrado} de los \textbf{periodos} ($T$) alrededor del Sol es \textbf{proporcional} al \textbf{cubo} de los \textbf{radios} medios de sus órbitas ($r$). Es decir:

\begin{equation}\label{GRAV_3LEYKEPLER}
	T^2 = k \cdot r^3
\end{equation}

Siendo k una \textbf{constante} igual para \textbf{todos los planetas}.

\subsection{Demostración de la Tercera Ley de Kepler}


\begin{equation}\label{GRAV_3KEPLER}
	\begin{rcases}
		\begin{aligned}
			G\frac{Mm}{r^2} = m\frac{v^2}{r} \Longrightarrow v^2= \frac{GM}{r}  \\
			T=\frac{2\pi r}{v}\Longrightarrow v^2=\frac{4\pi^2r^2}{T^2}
		\end{aligned}
	\end{rcases}
	\frac{GM}{r} = \frac{4\pi^2r^2}{T^2} \Longrightarrow T^2 = \frac{4\pi^2}{GM}r^3
\end{equation}



\subsection{El Momento Angular de los Planetas}

Cuando una partícula describe un movimiento curvilíneo, su estado de \textbf{movimiento} se caracteriza por su \textbf{momento angular} o momento cinético ($\vec{L}$):


\begin{equation}\label{GRAV_MOMENTOANGULAR}
	\vec{L} = \vec{r} \times \vec{p} = \vec{r} \times m\vec{v} \Longrightarrow {L} = r\cdot m\cdot v \cdot \sin{\alpha}
\end{equation}

\subsubsection{Teorema de Conservación del Momento Angular}

Fórmulas utilizadas: $\vec{M} = \vec{r}\times \vec{F}$, $\vec{L} = \vec{r} \times m\vec{v}$, $\vec{v} = \frac{d\vec{r}}{dt}$, $\vec{a} = \frac{d\vec{v}}{dt}$

\begin{equation}\label{GRAV_TEOREMACONSERVACIONL}
	\frac{d\vec{L}}{dt} = \frac{d}{dt} \left( \vec{r}\times m\vec{v}\right) = m \left(\cancelto{\vec{v}\times \vec{v} = 0}{\frac{d\vec{r}}{dt} \times \vec{v}} + \vec{r} \times \frac{d\vec{v}}{dt}\right) = \vec{r} \times \vec{F} = \vec{M}
\end{equation}

Como $\frac{d \vec{L}}{dt} = \vec{M}$, si $\vec{M} = 0 \Longrightarrow \vec{L} = \textnormal{cte.}$


\subsection{Ley de Gravitación Universal}

Conocida como la ley gravitacional de Newton, se expresa el \textbf{valor} de la \textbf{fuerza de atracción} entre dos masas. La \textbf{dirección} del vector es la recta que une las dos partículas. Las \textbf{fuerzas} que interactúan entre dos masas tienen el mismo \textbf{módulo} y \textbf{dirección} pero distinto sentido.


\begin{equation}\label{GRAV_LEYGRAVUNIVERSAL}
	\vec{F} = -G \frac{Mm}{r^2} \vec{u_r}
\end{equation}

Donde la Constante de Gravitación Universal es $G = \qty{6.67e-11}{\newton\meter\squared\per\kilo\gram\squared}$

\subsection{Campo Gravitatorio}

El campo gravitatorio se define como la \textbf{perturbación} que la masa produce en el \textbf{espacio} que le \textbf{rodea} por el hecho de tener masa.

\begin{enumerate}
	\item \textsc{\textbf{Líneas de Campo Gravitatorio}}:
	\begin{enumerate}
		\item Son \textbf{radiales} y van \textbf{dirigidos a la masa}.
		\item Son \textbf{tangentes} en cada punto al vector intensidad de campo y tienen su mismo sentido.
		\item No tienen \textbf{origen definido} (ya que el alcance del campo gravitatorio es infinito), pero terminan en puntos materiales denominados \textbf{sumideros} de campo.
		\item La \textbf{densidad} de líneas de campo es \textbf{proporcional} al módulo de la \textbf{intensidad} del campo.
		\item Las líneas de campo \textbf{no se pueden cortar}, ya que eso significaría que en un punto del espacio, el campo tendría dos valores distintos.
	\end{enumerate}
	\item \textsc{\textbf{Intensidad de Campo Gravitatorio ($\vec{g}$)}}
	
	\begin{equation}\label{GRAV_CAMPOGRAVITATORIO}
		\vec{g} = -G \frac{M}{r^2}\vec{u_r} \Longrightarrow g = G\frac{M}{r^2}
	\end{equation}
	
	
	\item \textsc{\textbf{Principio de Superposición}}
	\begin{enumerate}
		\item La intensidad del campo gravitatorio en un punto es la \textbf{suma vectorial} de los campos que crearía cada cuerpo aislado.
		\item De igual forma, la fuerza gravitatoria que siente una masa debido a otras masas será la \textbf{suma} de las fuerzas que cada una ejerzan:
		\begin{equation}\label{GRAV_SUPERPOS}
			\vec{F_T} = \Sigma F_i
		\end{equation}
	\end{enumerate}
\end{enumerate}

\begin{figure}[H]
	\centering
	\includegraphics[scale=2]{GRAV_CAMPOGRAV}
	\caption{Líneas de Campo Gravitatorio}
	\label{GRAV_CAMPOGRAV}
\end{figure}

\subsection{Fuerzas Conservativas y Energías }

Las \textbf{fuerzas} son \textbf{conservativas} cuando el trabajo que realiza dicha fuerza para trasladar una partícula de un punto A a otro B depende de los puntos inicial y final, pero \textbf{no} del camino \textbf{seguido}. En este caso, la \textbf{gravedad} es una fuerza \textbf{conservativa}.

\subsubsection{Energía Potencial Gravitatoria}

La energía potencial gravitatoria ($E_p$) es aquella que posee una masa $m$ por \textbf{encontrarse} bajo la \textbf{influencia} gravitatoria de otra masa $M$ u otras masas. También puede definirse como el \textbf{trabajo} que realizaría el campo gravitatorio para \textbf{trasladar} una masa m desde un punto hasta el \textbf{infinito}.

\begin{equation}\label{GRAV_ENERGIAPOTENCIAL}
	E_p = -G \frac{Mm}{r}
\end{equation}

\subsection{¿Por qué la \texorpdfstring{$E_p$}{Ep} también es \texorpdfstring{$m\cdot g\cdot h$}{mgh}?}

Con esta expresión, se asume que $h\ll R_t$

\begin{equation}
	\begin{split}
		\Delta E_p &= E_{pB} - E_{pA} = -G\frac{Mm}{R_t+h} + G\frac{Mm}{R_t} \\
		&= GMm \left(\frac{1}{R_t}-\frac{1}{R_t+h}\right) = \frac{GMmh}{R_t(R_t+h\simeq R_t)}
	\end{split}
\end{equation}

\begin{equation} 
	\Delta E_p = \frac{GM}{R^2_t}mh = mgh
\end{equation}


\subsubsection{Potencial Gravitatorio}

El potencial gravitatorio (V) en un punto se define como la \textbf{energía potencial gravitatoria} por \textbf{unidad} de masa en dicho punto:

\begin{equation}\label{GRAV_POTENCIAL}
	E_p = m\cdot V \Longrightarrow V = -G\frac{M}{r}
\end{equation}

\subsubsection{Trabajo}

Si $W > 0$, el trabajo lo realiza el \textbf{campo gravitatorio}. Si $W < 0$ el trabajo lo realiza una \textbf{fuerza exterior al campo}. Si $W = 0$ \textbf{no} se realiza \textbf{trabajo}. El trabajo para llevar una partícula de masa m desde un punto A hasta uno B será:

\begin{enumerate}
		\item $W_{A\rightarrow B} = -\Delta E_p$
		\item $W_{A\rightarrow B} = -m\Delta V$
\end{enumerate}

\subsection{Satélites}

\subsubsection{Velocidad Orbital}

Es la \textbf{velocidad necesaria} para que una masa $m$ (como un satélite, tanto natural como artificial) describa una \textbf{órbita circular} alrededor de \textbf{otra} $M$, para lo que la fuerza \textbf{centrípeta} debe ser \textbf{igual} que la fuerza \textbf{gravitatoria}.

\begin{equation}\label{GRAV_VORB}
	G\frac{M\cancel{m}}{r^{\cancel{2}}} = \cancel{m}\frac{v^2}{\cancel{r}} \Longrightarrow v_{\textnormal{orbital}} = \sqrt{\frac{GM}{r}}
\end{equation}

\subsubsection{Satélites Geoestacionarios}

Se llaman \textbf{satélites geosíncronos} a aquellos satélites cuyo periodo de revolución coincide con el de la Tierra: $T = 24\textnormal{h} = 86400\textnormal{s}$. Si además estos satélites están todo el rato \textbf{sobre el mismo punto} de la superficie terrestre (para lo que es necesario que su plano orbital coincida con el ecuador) entonces se denominan \textbf{satélites geoestacionarios}.

\subsubsection{Velocidad de Escape}

La velocidad de escape es la velocidad mínima que debe adquirir un cuerpo para escapar de la \textbf{atracción gravitatoria del planeta} en cuyas \textbf{proximidades} se encuentre. Esto significa que $E_p = 0$, $E_c = 0$ y $E_m = 0$.

\begin{equation}\label{GRAV_VESC}
	v_\textnormal{escape} = \sqrt{\frac{2GM}{r}} = \sqrt{2}\cdot v_\textnormal{orbital}
\end{equation}



\subsubsection{Energía Mecánica de un Satélite en Órbita}

La energía mecánica de un satélite en órbita, también denominada energía orbital, es la \textbf{suma} de la energía \textbf{potencial} y \textbf{cinética}:

\begin{equation}\label{GRAV_ENERGIAMECANICASATELITE}
	E_m = -G\frac{Mm}{2r}
\end{equation}

\subsubsection{Principio de Conservación de la Energía}

Es la energía que debemos \textbf{comunicarle} a un satélite para que \textbf{pase} de una \textbf{órbita} a otra, donde las energías mecánicas son distintas.

\begin{equation}\label{GRAV_CONSERVACIONENERGIA}
	W_{\textnormal{com}} + E_{p_A} + E_{c_A} = E_{c_B} + E_{p_B}
\end{equation}

De esta fórmula se puede \textbf{deducir} la fórmula de la \textbf{\textsc{energía de satelización}} (energía necesaria para poner un satélite en órbita):

\begin{equation}\label{GRAV_ENERGIASATELIZACION}
	E_{\textnormal{satelización}} = GMm \left( \frac{1}{r_A} - \frac{1}{2\cdot r_B} \right)
\end{equation}

\begin{mybox}{Conceptos Clave: Gravitación}
	\begin{itemize}
		\item \textbf{Distancia $r$:} En todas las fórmulas, $r$ es la distancia al \textbf{centro} del planeta. Si te dan la altura $h$, recuerda: $r = R_T + h$.
		\item \textbf{Signos:} La Energía Potencial ($E_p$) y el Potencial ($V$) son siempre \textbf{negativos} (el 0 está en el infinito).
	\end{itemize}
\end{mybox}



\subsection{Resumen de Fórmulas}
	
\begin{table}[H]
	\centering
	\renewcommand{\arraystretch}{2}
	\setlength{\tabcolsep}{6pt}
	\setlength{\aboverulesep}{4pt}
	\setlength{\belowrulesep}{4pt}
	\renewcommand{\tabularxcolumn}[1]{m{#1}}
	\rowcolors{2}{gray!10}{white}
	\begin{tabularx}{\textwidth}{
			>{\centering\arraybackslash}m{4.5cm}
			>{\centering\arraybackslash}X
			>{\centering\arraybackslash}m{3cm}}
		\toprule
		\rowcolor{gray!25}
		\textbf{Magnitud / Concepto} & \textbf{Fórmula} & \textbf{Unidades (SI)} \\
		\midrule
		
		\textsc{Ley de Gravitación Universal} & $\displaystyle \vec{F} = -G \frac{Mm}{r^2}\vec{u_r}$ & \unit{\newton} \\
		
		\textsc{Intensidad del Campo Gravitatorio} & $\displaystyle \vec{g} = -G\frac{M}{r^2}\vec{u_r}$ & \unit{\meter\per\second\squared} \\
		
		\textsc{Cte. Grav. Universal} & $\displaystyle G = \num{6.67e-11}$ & \unit{\newton\meter\squared\per\kilo\gram\squared} \\
		
		\textsc{Momento Angular} & $\displaystyle \vec{L} = \vec{r}\times m\vec{v}$ & \unit{\kilo\gram\meter\squared\per\second} \\
		
		\textsc{Energía Potencial Gravitatoria} & $\displaystyle E_p = -G\frac{Mm}{r} = mV$ & \unit{\joule} \\
		
		\textsc{Potencial Gravitatorio} & $\displaystyle V = -G\frac{M}{r}$ & \unit{\joule\per\kilo\gram} \\
		
		\textsc{Trabajo del Campo Gravitatorio} & $\displaystyle W_{A\rightarrow B} = -\Delta E_p = -m\Delta V$ & \unit{\joule} \\
		
		\textsc{Velocidad Orbital} & $\displaystyle v_{\textnormal{orbital}} = \sqrt{\frac{GM}{r}}$ & \unit{\meter\per\second} \\
		
		\textsc{Velocidad de Escape} & $\displaystyle v_{\textnormal{escape}} = \sqrt{\frac{2GM}{r}}$ & \unit{\meter\per\second} \\
		
		\textsc{Energía Mecánica en Órbita} & $\displaystyle E_m = -G\frac{Mm}{2r}$ & \unit{\joule} \\
		
		\textsc{Energía de Satelización} & $GMm\left(\frac{1}{r_A} - \frac{1}{2r_B}\right)$ & \unit{\joule} \\
		
		\textsc{Periodo Orbital} & $\displaystyle T = \frac{2\pi r}{v}$ & \unit{\second} \\
		
		\bottomrule
	\end{tabularx}
	\label{GRAV_TABLAFORMULARIO}
	\caption{Formulario de Gravedad}
\end{table}

% -------------- CAMPO ELÉCTRICO --------------

\section{Campo Eléctrico}

\subsection{Cargas Puntuales}

\subsubsection{Ley de Coulomb}

La fuerza de \textbf{atracción o repulsión} entre 2 cargas puntuales es directamente proporcional al producto de las cargas e \textbf{inversamente proporcional} al cuadrado de la distancia que las separa.

\begin{equation}\label{ELEC_COULOUMB}
	\vec{F} = k\frac{Qq}{r^2}\vec{u_r} \Longrightarrow F=k\frac{Qq}{r^2}
\end{equation}

Donde $k = \SI{9e9}{\newton\meter\squared\per\coulomb\squared}$ en el vacío, que está relacionada con $\varepsilon$ (permitividad eléctrica):

\[k = \frac{1}{4\pi\varepsilon}\]

\subsubsection{Líneas de Campo Eléctrico}

El campo eléctrico es la \textbf{perturbación} que genera un cuerpo por tener carga eléctrica.
\begin{enumerate}
	\item Si la carga es \textbf{positiva}, el campo eléctrico es de \textbf{repulsión}.
	\item Si la carga es \textbf{negativa}, el campo eléctrico es de \textbf{atracción}.
\end{enumerate}

El campo eléctrico, igual que el campo gravitatorio, es un \textbf{campo conservativo}.

\begin{figure}[H]
	\centering
	\includegraphics[scale=0.6]{ELEC_LINEAS}
	\caption{Líneas de Campo Magnético}
	\label{ELEC_LINEAS}
\end{figure}

\subsubsection{Intensidad del Campo Eléctrico}

La \textbf{intensidad} de campo eléctrico ($\vec{E}$) (también llamada simplemente campo eléctrico) en un punto se define como la \textbf{fuerza} que se ejerce por \textbf{unidad} de \textbf{carga} positiva situada en dicho punto:

\begin{equation} \label{ELEC_FCAMPOELECTRICO}
	\vec{F} = q\cdot\vec{E} 
\end{equation}

\begin{equation} \label{ELEC_CAMPOELECTRICO}
	\vec{E} = k\frac{Q}{r^2}\cdot\vec{u_r} \Longrightarrow E = k \frac{Q}{r^2}
\end{equation}


\subsubsection{Principio de Superposición}


\begin{equation}\label{ELEC_SUPERPOS}
	\vec{F_T} = \Sigma F_i
\end{equation}

\begin{equation}\label{ELEC_DISTANCIA}
	\Delta V = -\vec{E} \cdot \Delta\vec{r}
\end{equation}

\subsection{Energías y Fuerzas Conservativas}

La energía potencial eléctrica ($E_p$) es aquella que \textbf{posee} una carga por \textbf{encontrarse} bajo la influencia \textbf{eléctrica} de otra carga u otras cargas:

\begin{equation}\label{ELEC_EPOT}
	E_p = k\frac{Qq}{r}
\end{equation}

\subsubsection{Potencial Eléctrico (\texorpdfstring{$V$}{V})}

Es el trabajo \textbf{realizado} por el campo eléctrico para \textbf{trasladar} una \textbf{unidad} de carga desde un punto hasta el \textbf{infinito}.

\begin{equation}\label{ELEC_POENCIAL}
	V = \frac{E_p}{q} \Longrightarrow V = k\frac{Q}{r}
\end{equation}

\subsubsection{Trabajo (\texorpdfstring{$W$}{W})}

La fuerza \textbf{eléctrica} (al igual que la fuerza gravitatoria) es una \textbf{fuerza central}, ya que está dirigida hacia un punto. El campo eléctrico (igual que el campo gravitatorio) es un \textbf{campo conservativo}. El trabajo realizado por el campo \textbf{depende} solo de su estado \textbf{inicial} y \textbf{final}, no depende de la \textbf{trayectoria}.

\begin{equation}\label{ELEC_TRABAJO}
	W_{A\rightarrow B} = E_{pA} - E_{pB} \Longrightarrow W_{A\rightarrow B} = -q(V_B - V_A)
\end{equation}

\clearpage

\subsection{Campos Eléctricos Uniformes}

\subsubsection{Líneas de Campo}

\begin{wrapfigure}{R}{0.5\textwidth}
	\centering
	\includegraphics[scale=0.15]{ELEC_LINEASCAMPOSUNIFORMES}
	\caption{Líneas de Campo Eléctrico en un Campo Uniforme}
	\label{ELEC_LINEASCAMPOUNIFORME}
\end{wrapfigure}

Las líneas de campo de un campo eléctrico \textbf{uniforme} son \textbf{líneas} de campo \textbf{paralelas}.

\subsubsection{Movimiento de Carga en Campo Uniforme}

Cuando trabajamos con un campo eléctrico uniforme, al ser \textbf{constante}, aplicaremos también la \textbf{segunda ley de Newton}, $\Sigma F = ma$, siendo esta \textbf{aceleración} también \textbf{uniforme}.

Por lo tanto, podremos utilizar en este caso las \textbf{fórmulas del MRUA}:

\begin{equation}\label{ELEC_MOVUNIFORME}
	\begin{rcases}
		\begin{aligned}
			x = x_0 + v_0 t + \frac{1}{2}at^2  \\
			v = v_0 + at \\
			v^2 = v_o^2 + 2a\Delta x
		\end{aligned}
	\end{rcases}
\end{equation}

Podemos resolverlo también mediante el \textbf{Principio de Conservación de la Energía Mecánica}, al ser la fuerza eléctrica \textbf{conservativa}:

\begin{equation}\label{ELEC_CONSERVACIONENERGIA}
	\Delta E_m = 0 \Longrightarrow \Delta E_c = -\Delta E_p
\end{equation}

\subsubsection{Campo: Láminas Infinitas}

\begin{figure}[H]
	\centering
	\includegraphics[scale=0.05]{ELEC_CAMPOSLAMINASINFINITAS}
	\caption{Campo Eléctrico Creado por Láminas Infinitas}
	\label{ELEC_CAMPOLAMINASINFINITAS}
\end{figure}

\begin{equation}\label{ELEC_LAMINASINFINITAS}
	E = \frac{\sigma}{\varepsilon}
\end{equation}

Siendo $\sigma$ las cargas por superficie $\frac{\textnormal{cargas ($C$)}}{\textnormal{superficie ($m^2$)}}$ y $\varepsilon$ depende del material.

\subsubsection{Campo: Hilo Infinito}

Las líneas de campo salen radialmente del hilo (suponiendo $\lambda$ positivo).

\begin{figure}[H]
	\centering
	\includegraphics[scale=0.3]{ELEC_CAMPOSHILOINFINITO}
	\caption{Campo Eléctrico Creado por Un Hilo Infinito}
	\label{ELEC_CAMPOHILOINFINITO}
\end{figure}

\begin{equation}\label{ELEC_HILOINFINITO}
	E = 2k\frac{\lambda}{r} = \frac{\lambda}{2\varepsilon\pi r}
\end{equation}

Siendo $\lambda$ las cargas por longitud $\frac{\textnormal{cargas ($C$)}}{\textnormal{longitud ($m$)}}$ y $\varepsilon$ depende del material.

\begin{mybox}{El Electronvoltio}
	Un Electronvoltio se define como la energía que tiene un electrón sometido a una diferencia de potencial de \SI{1}{\volt}.
	
	\[
		\qty{1}{\electronvolt} = \qty{1.602e-19}{\joule}
	\]
\end{mybox}

	\begin{mybox}{Conceptos Clave: Campo Eléctrico}
		\begin{itemize}
			\item \textbf{Vectores vs Escalares:}
			\begin{itemize}
				\item Fuerza ($\vec{F}$) y Campo ($\vec{E}$) son \textbf{vectores}
				\item Potencial ($V$) y Energía ($E_p$) son \textbf{escalares}
			\end{itemize}
			\item \textbf{Signo de la carga:} En las fórmulas escalares ($V, E_p$), \textbf{incluye} el signo de la carga. En las vectoriales, usa el signo para determinar el sentido del vector.
		\end{itemize}
	\end{mybox}

\clearpage

\subsection{Resumen de Fórmulas}

\begin{table}[H]
	\centering
	\renewcommand{\arraystretch}{1.5}
	\setlength{\tabcolsep}{6pt}
	\setlength{\aboverulesep}{0pt}
	\setlength{\belowrulesep}{0pt}
	\renewcommand{\tabularxcolumn}[1]{m{#1}}
	\rowcolors{2}{gray!10}{white}
	\begin{tabularx}{\textwidth}{
			>{\centering\arraybackslash}m{4.5cm}
			>{\centering\arraybackslash}X
			>{\centering\arraybackslash}m{3cm}}
		\toprule
		\rowcolor{gray!25}
		\textbf{Magnitud / Concepto} & \textbf{Fórmula} & \textbf{Unidades (SI)} \\
		\midrule
		
		\textsc{Ley de Coulomb} & $\displaystyle \vec{F} = k\frac{Qq}{r^2}\vec{u_r}$ & \unit{\newton} \\
		
		\textsc{Constante de Coulomb} & $\displaystyle k = \SI{9e9}{} = \frac{1}{4\pi\varepsilon}$ & \unit{\newton\meter\squared\per\coulomb\squared} \\
		
		\textsc{Intensidad del Campo Eléctrico (Definición)} & $\displaystyle \vec{F} = q\vec{E}$ & \unit{\newton} \\
		
		\textsc{Campo Eléctrico Creado por una Carga Puntual} & $\displaystyle \vec{E} = k\frac{Q}{r^2}\vec{u_r}$ & \unit{\newton\per\coulomb} \\
		
		\textsc{Principio de Superposición} & $\displaystyle \vec{F}_T = \sum_i \vec{F}_i$ & \unit{\newton} \\
		
		\textsc{Energía Potencial Eléctrica} & $\displaystyle E_p = k\frac{Qq}{r}$ & \unit{\joule} \\
		
		\textsc{Potencial Eléctrico} & $\displaystyle V = \frac{E_p}{q} = k\frac{Q}{r}$ & \unit{\volt} = \unit{\joule\per\coulomb} \\
		
		\textsc{Trabajo del Campo Eléctrico} & $\displaystyle W_{A\rightarrow B} = E_{pA} - E_{pB} = -q(V_B - V_A)$ & \unit{\joule} \\
		
		\textsc{Campo Eléctrico Uniforme} & $\displaystyle E = \frac{F}{q}$ & \unit{\newton\per\coulomb} \\
		
		\textsc{Campo Creado por Láminas Infinitas} & $\displaystyle E = \frac{\sigma}{\varepsilon}$ & \unit{\newton\per\coulomb} \\
		
		\textsc{Campo Creado por un Hilo Infinito} & $\displaystyle E = 2k\frac{\lambda}{r} = \frac{\lambda}{2\pi\varepsilon r}$ & \unit{\newton\per\coulomb} \\
		
		\textsc{Relación entre $V$ y $E$} & $\Delta V = -\vec{E} \cdot \Delta\vec{r}$ & --- \\
		
		\textsc{Electron-voltio} & $\displaystyle \qty{1}{\electronvolt} = \num{1.602e-19}$ & \unit{\joule} \\
		
		\bottomrule
	\end{tabularx}
	\caption{Formulario del Campo Eléctrico}
	\label{ELEC_TABLAFORMULARIO}
\end{table}

% -------------- CAMPO MAGNÉTICO --------------

\section{Campo Magnético}

El campo \textbf{magnético} es la \textbf{perturbación} que genera un \textbf{imán} o \textbf{cargas} en \textbf{movimiento} (corrientes eléctricas). El campo \textbf{magnético}, a diferencia de los campos gravitatorios y eléctricos, es un campo \textbf{no conservativo}, ya que el trabajo \textbf{sí} depende de la \textbf{trayectoria}.

La \textbf{intensidad} de campo magnético $B$ es una magnitud \textbf{vectorial}, cuya unidad en el \textbf{Sistema Internacional} es el \textbf{Tesla} (\unit{\tesla}). El \textbf{tesla} es una unidad muy \textbf{grande}. Por ejemplo, el campo magnético terrestre es del orden de \SI{e-5}{\tesla}. Por eso a veces se usa una \textbf{unidad menor}, denominada \textbf{Gauss} (G): \SI{1}{G} = \SI{e-4}{\tesla}.


\begin{figure}[H]
	\centering
	\begin{subfigure}{0.45\textwidth}
		\centering
		\includegraphics[scale=0.3]{MAGN_LINEASHILO}
		\caption{\textbf{Hilo}}
		\label{MAGN_LINEASHILO}
	\end{subfigure}
	\begin{subfigure}{0.45\textwidth}
		\centering
		\includegraphics[scale=0.2]{MAGN_LINEASIMAN}
		\caption{\textbf{Imán}}
		\label{MAGN_LINEASIMAN}
	\end{subfigure}
	\caption{Líneas de Campo Creados por un Hilo y por un Imán}
	\label{MAGN_LINEASDECAMPO}
\end{figure}

\subsection{Ley de Lorentz}

Cuando un cuerpo cargado penetra con una \textbf{velocidad} $\vec{v}$ en una región del espacio en la que existe un \textbf{campo magnético} $\vec{B}$, se ve sometido a una \textbf{fuerza magnética} $\vec{F_m}$:

\begin{equation}\label{MAGN_LORENTZ}
	\vec{F_m} = q\cdot \vec{v}\times \vec{B} \Longrightarrow F_m = \left| q \right|\cdot v\cdot B\sen{\alpha}
\end{equation}

Siendo $\alpha$ el \textbf{ángulo} que forman $v$ y $B$. La unidad de $B$ son los Teslas (\unit{\tesla}), que son: \unit{\newton\per\coulomb\per\meter\second}.

\begin{enumerate}
	\item Si $\vec{v}$ y $\vec{B}$ son \textbf{\textsc{paralelos}}: $\alpha = 0^\circ \Longrightarrow F_m = 0$, por lo tanto, la carga describirá un \textbf{MRU}.
	
	\item Si $\vec{v}$ y $\vec{B}$ son \textbf{\textsc{perpendiculares}}, la \textbf{fuerza} será \textbf{máxima} y la carga describirá un \textbf{MCU}.
	
	\item En el \textbf{resto} de casos, la carga describirá un \textbf{movimiento helicoidal} (véase \cref{MAGN_MOVIMIENTO}).
\end{enumerate}

\subsection{Características del Movimiento}

La partícula que penetra con \textbf{velocidad perpendicular} al campo magnético. El \textbf{radio} de la circunferencia $r$ que describe una partícula cargada \textbf{depende} de su \textbf{velocidad}. El \textbf{periodo} ($T$) del movimiento circular lo podemos \textbf{calcular} también.

Para las partículas que penetran con un \textbf{ángulo} $\alpha$ \textbf{cualquiera} con el campo magnético se pueden \textbf{descomponer} en \textbf{ejes} \textbf{perpendiculares} y \textbf{paralelos} al campo magnético.

El \textbf{paso} ($d$ en la figura \ref{MAGN_MOVIMIENTO}) de la hélice en el caso en el que la partícula penetra con un \textbf{ángulo} $\alpha$ \textbf{cualquiera} con el campo magnético es la distancia que \textbf{recorrería} al girar una vuelta completa. Se calcula como la \textbf{velocidad paralela} al campo magnético por su \textbf{periodo}: $d = v_{\parallel}\cdot T$. 

\begin{equation}\label{MAGN_RADIOMOVIMIENTO}
	r = \frac{m\cdot v}{q\cdot B}
\end{equation}

\begin{equation}\label{MAGN_PERIODOMOVIMIENTO}
	T = \frac{2\pi r}{v} = \frac{2\pi m}{qB}
\end{equation}

\begin{equation}\label{MAGN_DESCOMPOSICIÓN}
	\vec{v} = \vec{v}_{\perp} + \vec{v}_{\parallel} = v_{\perp}\cdot \vec{u_{\perp}} + v_{\parallel}\cdot \vec{u_{\parallel}}
\end{equation}

\begin{figure}[H]
	\centering
	\includegraphics[scale=0.5]{MAGN_MOVIMIENTO}
	\caption{Movimiento Descrito por una Carga con una Velocidad bajo un Ángulo}
	\label{MAGN_MOVIMIENTO}
\end{figure}

\subsection{Cálculo de un Producto Vectorial} \label{MAGN_EPIGRAFEPRODUCTOVECTORIAL}

Para obtener el \textbf{vector fuerza} magnética $\vec{F}$ a partir de las \textbf{componentes} de la \textbf{velocidad} $\vec{v} = v_x\vec{i} + v_y\vec{j} + v_z\vec{k}$ y del \textbf{campo magnético} $\vec{B} = B_x\vec{i} + B_y\vec{j} + B_z\vec{k}$, utilizamos la regla del determinante:

\begin{equation}\label{MAGN_PRODUCTOVECTORIAL}
	\vec{F} = q (\vec{v} \times \vec{B}) = q 
	\begin{vmatrix}
		\vec{i} & \vec{j} & \vec{k} \\
		v_x & v_y & v_z \\
		B_x & B_y & B_z
	\end{vmatrix}
\end{equation}

Desarrollando el determinante:

\begin{equation}\label{MAGN_PRODUCTOVECTORIALDEVELOPADO}
	\vec{F} = q \left[ (v_y B_z - v_z B_y)\vec{i} - (v_x B_z - v_z B_x)\vec{j} + (v_x B_y - v_y B_x)\vec{k} \right]
\end{equation}

También se puede usar la regla de la mano derecha o del sacacorchos:

\begin{figure}[H]
	\centering
	\begin{subfigure}{0.45\textwidth}
		\centering
		\includegraphics[scale=0.03]{MAGN_MANODERECHA.png}
		\caption{Regla de la Mano Derecha}
		\label{MAGN_MANODERECHA}
	\end{subfigure}
	\begin{subfigure}{0.45\textwidth}
		\centering
		\includegraphics[scale=1]{MAGN_SACACORCHOS.png}
		\caption{Regla del Sacacorchos}
		\label{MAGN_SACACORCHOS}
	\end{subfigure}
	\caption{Reglas de la Mano Derecha y del Sacacorchos}
	\label{MAGN_MANODERECHASACACORCHOS}
\end{figure}

La regla de la \textbf{mano derecha} (o del \textbf{sacacorchos}) determina \textbf{sentidos vectoriales}. Para el sacacorchos, giramos de $\vec{v}$ a $\vec{B}$. Para la mano derecha, véase la figura \ref{MAGN_MANODERECHA}. 

\subsection{El Selector de Velocidades}

El selector de velocidades utiliza campos \textbf{eléctrico} y \textbf{magnético} cruzados. Si las \textbf{partículas cargadas} entran con cierta \textbf{velocidad}, las fuerzas se \textbf{cancelan}. Esto permite que sigan una \textbf{trayectoria rectilínea} y \textbf{atraviesen} el dispositivo \textbf{sin desviarse}.

\begin{figure}[H]
	\centering
	\includegraphics[scale=0.2]{MAGN_SELECTORDEVELOCIDADES}
	\caption{Selector de Velocidades}
	\label{MAGN_SELECTORDEVELOCIDADES}
\end{figure}

Para seleccionar las partículas que tienen una \textbf{cierta velocidad}, la fuerza eléctrica y magnética han de ser \textbf{iguales}:

\begin{equation}\label{MAGN_SELECTORFORMULAS}
	\vec{F_E} = \vec{F_B} \Longrightarrow E = vB \Longrightarrow v = \frac{E}{B}
\end{equation}

\subsection{Espectrómetro de Masas}

Es un \textbf{dispositivo} que empleado para \textbf{separar partículas} cargadas que poseen \textbf{distinta relación} ($\frac{\textnormal{carga}}{\textnormal{masa}}$). El espectrómetro de masas consta básicamente de un \textbf{selector de velocidades} (que permite seleccionar las partículas con una determinada velocidad) seguido de una \textbf{zona} en la que se establece un \textbf{campo magnético}. En esta zona, la partícula \textbf{cargada} describe una trayectoria \textbf{circular}, y un a placa \textbf{fotográfica} recoge el \textbf{impacto} de las partículas después de \textbf{describir una semicircunferencia}. De esta forma podemos medir el \textbf{radio} de curvatura y calcular la relación $\frac{\textnormal{carga}}{\textnormal{masa}}$.

\begin{figure}[H]
	\centering
	\includegraphics[scale=0.4]{MAGN_ESPECTROMETRO}
	\caption{Espectrómetro de Masas}
	\label{MAGN_ESPECTROMETRO}
\end{figure}

\subsection{Ciclotrón}

Un ciclotrón es un \textbf{acelerador de partículas} cargadas que después suelen ser utilizadas para producir \textbf{reacciones nucleares} o para \textbf{obtener información} sobre otros núcleos.

\begin{figure}[H]
	\centering
	\includegraphics[scale=0.45]{MAGN_CICLOTRON}
	\caption{Ciclotrón}
	\label{MAGN_CICLOTRON}
\end{figure}

\begin{enumerate}
	\item Un ciclotrón tiene dos \textbf{partes} llamadas ``\textit{Des}'' (por su forma). Son recipientes \textbf{semicirculares} al \textbf{vacío}, colocados \textbf{perpendicularmente} a un \textbf{campo} $B$. Las partículas describen \textbf{trayectorias} circulares de radio \textbf{creciente}. Las dos ``$Ds$'', $D_1$ y $D_2$, están separadas cierta \textbf{distancia}.

	\item En \textbf{ese espacio} se \textbf{acelera} la partícula mediante una \textbf{diferencia de potencial}, \textbf{aumentando} su radio, hasta que alcanza un \textbf{radio máximo} denominado \textbf{radio de extracción}.

	\item El \textbf{periodo} es \textbf{independiente} de la velocidad de la partícula y de su \textbf{radio}, por lo que será \textbf{constante} en el \textbf{ciclotrón}. Tras una serie de \textbf{vueltas}, la partícula \textbf{alcanzará} una energía \textbf{cinética} máxima con la que saldrá del \textbf{ciclotrón}.
\end{enumerate}

\begin{equation}\label{MAGN_CICLOTRONPERIODO}
	\begin{rcases}
		\begin{aligned}
			\left| q \right|Bv = m\frac{v^2}{r} \Longrightarrow \frac{r}{v} = \frac{m}{\left| q \right|B}
			\\
			T=\frac{2\pi r}{v}
		\end{aligned}
	\end{rcases}
T = \frac{2\pi m}{\left| q \right|B}
\end{equation}


La \textbf{energía cinética máxima} que alcanza la partícula depende del radio máximo del ciclotrón ($R$) y del campo magnético ($B$):

\begin{equation}\label{MAGN_ECINETICAMAX}
	E_{\textnormal{c,max}} = \frac{q^2 B^2 R^2}{2m}
\end{equation}

\clearpage

\subsection{Efecto de un Campo Magnético sobre un Hilo de Corriente}

Si un \textbf{hilo} que transporta una \textbf{corriente eléctrica} se encuentra en un \textbf{campo magnético}, experimenta una \textbf{fuerza magnética} que podemos deducir a partir de la Ley de Lorentz. Recordemos que la \textbf{intensidad} se define como $I = \frac{dq}{dt}$. Sustituyendo en la expresión diferencial de la fuerza:

\begin{equation}
d\vec{F}_B = dq \cdot \vec{v} \times \vec{B}
\quad\Longrightarrow\quad
d\vec{F}_B = I \cdot dt \cdot \vec{v} \times \vec{B}
\end{equation}

Como $\vec{v} \cdot dt = d\vec{\ell}$, siendo $d\vec{\ell}$ un elemento de \textbf{longitud} en la dirección de la corriente:

\begin{equation}
d\vec{F}_B = I \cdot d\vec{\ell} \times \vec{B}
\end{equation}

\begin{equation}\label{MAGN_FUERZAHILOFORMULA}
\vec{F}_B = I \cdot \vec{\ell} \times \vec{B}
\end{equation}

Y su módulo queda:

\begin{equation}\label{MAGN_FUERZAHILOMODULO}
F = I \cdot \ell \cdot B \cdot \sin\alpha
\end{equation}

En esta fórmula, $\vec{\ell}$ es un vector cuya \textbf{dirección} y \textbf{sentido} coinciden con los de la corriente, y cuyo módulo es la longitud del tramo considerado del hilo. Además, la \textbf{dirección} y \textbf{sentido} de la fuerza pueden deducirse mediante el cálculo del producto vectorial o la regla de la \textbf{mano derecha} (epígrafe \ref{MAGN_EPIGRAFEPRODUCTOVECTORIAL}).

\begin{figure}[H]
	\centering
	\includegraphics[scale=0.7]{MAGN_FUERZAHILO}
	\caption{Fuerzas que actúan sobre un hilo de corriente}
	\label{MAGN_FUERZAHILO}
\end{figure}

\clearpage

\subsection{Campo Magnético creado por un Hilo de Corriente}

Un \textbf{hilo} de corriente por el que pasa una \textbf{intensidad} $I$ crea un \textbf{campo magnético} en sus \textbf{proximidades}. Para un punto $P$ cuya distancia más corta al hilo sea $x$, el módulo de la \textbf{intensidad de campo} es:

\begin{equation}\label{MAGN_CAMPOCREADOHILOFORMULA}
	B = \frac{\mu}{2\pi} \cdot \frac{I}{x}
\end{equation}

Las líneas de campo son circunferencias \textbf{centradas} en el hilo, que se encuentran en el plano \textbf{perpendicular} al \textbf{hilo} y su sentido viene dado por la \textbf{regla del sacacorchos} (figura \ref{MAGN_SACACORCHOS}).

\begin{figure}[H]
	\centering
	\includegraphics[scale=0.5]{MAGN_CAMPOCREADOHILO}
	\caption{Campo magnético creado por un hilo de corriente}
	\label{MAGN_CAMPOCREADOHILO}
\end{figure}

\clearpage

\subsection{Acciones entre corrientes}

Si circulan \textbf{corrientes} por \textbf{varios hilos paralelos}, se dan \textbf{interacciones magnéticas} entre ellas. Una situación en la que esto puede ocurrir es en los cables que transportan corriente en la red eléctrica. Supongamos que dos hilos paralelos, de longitud $L$ y separados una distancia $d$. Cada uno de ellos creará un \textbf{campo magnético} $B$ que podemos calcular:

Campo que crea la \textbf{corriente 1} sobre \textbf{2}:

\begin{equation}\label{MAGN_CAMPOCREADOHILO1}
	B_{1} = \frac{\mu}{2\pi} \cdot \frac{I_1}{d}
\end{equation}

Fuerza que sufre la \textbf{corriente 2}:

\begin{equation}\label{MAGN_FUERZAHILO2}
	F_{12} = I_2 \cdot \ell \cdot B_1 
\end{equation}

Sustituyendo \ref{MAGN_CAMPOCREADOHILO1} en \ref{MAGN_FUERZAHILO2}:

\begin{equation}
	F_{12} = I_2 \cdot \ell \cdot \frac{\mu}{2\pi} \cdot \frac{I_1}{d}
\end{equation}

La fuerza se suele medir por \textbf{unidad de longitud} ($\frac{F}{\ell}$).

\begin{equation}\label{MAGN_FUERZAPORUNIDADA}
	\frac{F_{12}}{\ell} = \frac{\mu}{2\pi} \cdot \frac{I_1 I_2}{d}
\end{equation}

\begin{figure}[H]
	\centering
	\begin{subfigure}{0.45\textwidth}
		\centering
		\includegraphics[scale=0.25]{MAGN_HILOSINTERACCION.png}
		\caption{Mismo Sentido}
		\label{MAGN_MISMOSSENTIDO}
	\end{subfigure}
	\begin{subfigure}{0.45\textwidth}
		\centering
		\includegraphics[scale=0.25]{MAGN_HILOSINTERACCION2.png}
		\caption{Sentido Contrario}
		\label{MAGN_SENTIDONOTRAS}
	\end{subfigure}
	\caption{Interacciones entre hilos de corriente}
	\label{MAGN_HILOSINTERACCION}
\end{figure}

\subsection{Definición de Amperio}

Hasta el año 2019, el \textbf{amperio} (la intensidad de la corriente), se definía sobre la base de la fuerza de interacción magnética entre dos conductores rectilíneos paralelos. Si $I_1 = I_2 = \SI{1}\ampere$ y $d = \SI{1}\metre$, dado que $\mu = 4 \pi \cdot 10^{-7} \unit{\newton\per\ampere\squared}$:

\begin{equation}
	\frac{F_{12}}{\ell} = \frac{\mu}{2\pi} \cdot \frac{1 \unit{\ampere} \cdot 1 \unit{\ampere}}{1 \unit{\metre}} = 2 \cdot 10^{-7} \unit{\newton\per\metre}
\end{equation}

Así, un amperio internacional o Ampère ($A$) era la intensidad de corriente eléctrica que debía circular por dos conductores rectilíneos, paralelos e indefinidos para que, separados por una distancia de $1 \unit{\metre}$, ejerciera una fuerza entre ellos de $2 \cdot 10^{-7} \unit{\newton}$ por cada metro de conductor. A partir de 2019, se definió el amperio a partir de la carga elemental del electrón.

\begin{figure}[H]
	\centering
	\includegraphics[scale=0.3]{MAGN_AMPERIO.png}
	\caption{Definición del amperio hasta 2019}
	\label{MAGN_AMPERIO}
\end{figure}



\begin{mybox}{Conceptos Clave: Campo Magnético}
	\begin{itemize}
		\item \textbf{Trabajo Nulo:} La fuerza magnética \textbf{nunca} realiza trabajo ($W=0$) porque es siempre perpendicular a la velocidad.
		\item \textbf{Energía Cinética:} Como no hay trabajo, la fuerza magnética \textbf{no cambia} la rapidez (módulo de la velocidad), solo curva la trayectoria. $E_c = \textnormal{cte}$.
	\end{itemize}
\end{mybox}

\clearpage

\subsection{Resumen de Fórmulas}

\begin{table}[H]
	\centering
	\renewcommand{\arraystretch}{1.5}
	\setlength{\tabcolsep}{6pt}
	\setlength{\aboverulesep}{0pt}
	\setlength{\belowrulesep}{0pt}
	\renewcommand{\tabularxcolumn}[1]{m{#1}}
	\rowcolors{2}{gray!10}{white}
	\begin{tabularx}{\textwidth}{
			>{\centering\arraybackslash}m{4.5cm}
			>{\centering\arraybackslash}X
			>{\centering\arraybackslash}m{3cm}}
		\toprule
		\rowcolor{gray!25}
		\textbf{Magnitud / Concepto} & \textbf{Fórmula} & \textbf{Unidades (SI)} \\
		\midrule
		
		\textsc{Ley de Lorentz (Vectorial)} & $\displaystyle \vec{F}_m = q(\vec{v}\times\vec{B})$ & \unit{\newton} \\
		
		\textsc{Ley de Lorentz (Módulo)} & $\displaystyle F_m = |q|vB\sin\alpha$ & \unit{\newton} \\
		
		\textsc{Radio de la Trayectoria} & $\displaystyle r = \frac{mv}{|q|B}$ & \unit{\meter} \\
		
		\textsc{Periodo (Partícula / Ciclotrón)} & $\displaystyle T = \frac{2\pi m}{|q|B}$ & \unit{\second} \\
		
		\textsc{Frecuencia (Ciclotrón)} & $\displaystyle f = \frac{|q|B}{2\pi m}$ & \unit{\hertz} o \unit{\per\second} \\
				
		\textsc{Paso de la Hélice} & $\displaystyle p = v_{\parallel} \cdot T$ & \unit{\meter} \\
		
		\textsc{Selector de Velocidades} & $\displaystyle v = \frac{E}{B}$ & \unit{\meter\per\second} \\
				
		\textsc{Energía Cinética Máxima (Ciclotrón)} & $\displaystyle E_{\textnormal{c,max}} = \frac{q^2 B^2 R^2}{2m}$ & \unit{\joule} \\

		\textsc{Fuerza sobre un hilo de corriente} & $\displaystyle \vec{F} = I\vec{\ell} \times \vec{B}$ & \unit{\newton} \\

		\textsc{Campo magnético creado por un hilo de corriente} & $\displaystyle B = \frac{\mu}{2\pi} \cdot \frac{I}{x}$ & \unit{\tesla} \\

		\textsc{Fuerza entre corrientes} & $\displaystyle \frac{F_{12}}{\ell} = \frac{\mu}{2\pi} \cdot \frac{I_1 I_2}{d}$ & \unit{\newton\per\meter} \\
		
		\bottomrule
	\end{tabularx}
	\caption{Formulario de Campo Magnético}
	\label{MAGN_TABLAFORMULARIO}
\end{table}

\end{document}