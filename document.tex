% preámbulo

\documentclass[titlepage,12pt]{article} %% tamaño del texto
\usepackage[spanish]{babel} %% idioma
\usepackage[a4paper]{geometry} %% márgenes
\usepackage[utf8]{inputenc} %% soporte para UTF-8
\usepackage[T1]{fontenc} %% mejora la codificación de salida
\renewcommand{\theenumi}{\roman{enumi}} %% números romanos para las enumeraciones (comentar si no es necesario)
\usepackage{graphicx} %% imágenes y gráficos
\graphicspath{ {FIGURES/} }
\usepackage{siunitx}
\sisetup{output-decimal-marker = {,}}
\usepackage{microtype}
\hyphenpenalty=1000 %% evitar separar las palabras con guión
\interlinepenalty=1000000 %% evitar saltos de línea entre párrafos
\usepackage[hidelinks]{hyperref} %% pone los hipervínculos en figuras y tablas de contenidos

\usepackage{bm} % usar la negrita en los símbolos matemáticos
\usepackage{color,soul}
\setlength{\parindent}{20pt} % sangrado
\usepackage{wrapfig}
\usepackage{caption}
\usepackage{subcaption}
\usepackage{pgfplots}

\usepackage[]{titlesec} %cambiar espacios en secciones
\titlespacing*{\section}{0pt}{1.2ex}{1.2ex}
\titlespacing*{\subsection}{0pt}{1ex}{1ex}
\titlespacing*{\subsubsection}{0pt}{0.5ex}{0.5ex}
\usepackage{booktabs}   % Líneas bonitas
\usepackage[table]{xcolor}  % Colores para la tabla
\usepackage{tabularx}   % Ajuste automático al ancho
\usepackage{float}
\addto\captionsspanish{\renewcommand{\tablename}{Tabla}}
\usepackage{colortbl}
\usepackage{lipsum}
\usepackage[most]{tcolorbox}
\tcbuselibrary{skins,breakable}
\newtcolorbox{mybox}[2][]{breakable,sharp corners, skin=enhancedmiddle jigsaw,parbox=false,
	boxrule=0mm,leftrule=2mm,boxsep=0mm,arc=0mm,outer arc=0mm,attach title to upper,
	after title={.\ }, coltitle=black,colback=gray!10,colframe=black, title={#2},
	fonttitle=\bfseries,#1}

\titleformat*{\section}{\Huge\bfseries\scshape}
\titleformat*{\subsection}{\Large\bfseries\scshape}
\titleformat*{\subsubsection}{\large\bfseries}
\titleformat*{\paragraph}{\large\bfseries}
\titleformat*{\subparagraph}{\large\bfseries}
%\setcounter{tocdepth}{2}
\usepackage{array}


% comentar si no se quieren puntos en la tabla de contenidos
\usepackage[makeroom]{cancel}
\usepackage{tocloft} %% forzar los puntos
\renewcommand{\cftdot}{·} %% definir el carácter de relleno (puede ser otro)
\renewcommand{\cftsecleader}{\cftdotfill{\cftdotsep}} %% aplicarlo a las secciones
\renewcommand{\cftsubsecleader}{\cftdotfill{\cftdotsep}} %% aplicarlo a las sub-secciones
\usepackage{mathtools}
\usepackage{multicol}
% título, autor, fecha
\usepackage{cleveref}

\title{\textsc{\huge{\textbf{Física 2\textsuperscript{o} Bachillerato}}}}
\author{Manuel Sánchez Abián}
\date{2025-2026}

% documento

\pgfplotsset{compat=1.18}


\begin{document}

\maketitle
\thispagestyle{empty}
\setcounter{page}{0}
\thispagestyle{empty}
\setcounter{page}{0}
\tableofcontents
\addtocontents{toc}{\protect\thispagestyle{empty}}
\thispagestyle{empty}
\clearpage
\setcounter{page}{1}


% -------------- CAMPO GRAVITATORIO --------------

\include{SECTIONS/GRAV}

% -------------- CAMPO ELÉCTRICO --------------

\include{SECTIONS/ELECT}

% -------------- CAMPO MAGNÉTICO --------------

\include{SECTIONS/MAGN}

% -------------- INDUCCIÓN ELECTROMAGNÉTICA --------------

\section{Inducción Electromagnética}

\subsection{El Flujo Magnético}

El \textbf{flujo magnético} ($\phi$) se define, en una \textbf{superficie}, como el número de \textbf{líneas de inducción} que la \textbf{atraviesan}. Este número número de \textbf{líneas por superficie} perpedincular a las mismas indica la intensidad del campo, es decir:

\begin{equation}\label{INDUCC_FLUJO1}
	\phi = \vec{B} \cdot \vec{S} \Longrightarrow \phi = B \cdot S \cdot \cos \varphi 
\end{equation}

El flujo magnético se mide en \textbf{Wéber} (\unit{\weber}).

\begin{figure}[H]
	\centering
	\includegraphics[scale=0.2]{INDUCC_SITUACIONESFLUJO}
	\caption{Situaciones del Flujo Magnético}
	\label{INDUCC_SITUACIONESFLUJO}
\end{figure}

\subsection{Ley de Faraday--Lenz}

La Ley de Faraday--Lenz establece que cuando el \textbf{flujo magnético} que atraviesa un circuito varía, aparece una \textbf{fuerza electromotriz} (\textit{fem}) inducida. Su valor viene dado por:

\begin{equation}\label{INDUCC_LEYFARADAYLENZ}
\varepsilon = -\frac{d\phi}{dt}
\end{equation}

El signo \textbf{negativo} de la fórmula \ref{INDUCC_LEYFARADAYLENZ} indica que la \textbf{corriente} inducida \textbf{siempre se opone} al cambio de flujo que la produce (sentido de Lenz). La \textbf{fem} \(\varepsilon\) es la energía por unidad de carga que surge en el circuito debido a esa variación de flujo y se mide en voltios (\unit{\volt}). \textbf{Ejemplo:} Si un imán se acerca a una espira, el flujo magnético aumenta y aparece una corriente cuyo campo se opone a ese acercamiento. Si el imán se aleja, el flujo disminuye y la corriente inducida cambia de sentido para intentar mantener el flujo anterior.

\clearpage

\begin{mybox}{Ejercicio Ejemplo}
	\textbf{Una espira triangular de $\SI{4}{\meter}$ de lado se desplaza a $\SI{2}{\meter\per\second}$ hacia una región donde hay un campo magnético $\vec{B}$ perpendicular al plano de la espira, tal y como se indica en la figura \ref{INDUCC_EJEMPLO1}. En $t = \SI{0}{\second}$ la espira está a $\SI{2}{\meter}$ de la región.}

	\begin{figure}[H]
		\centering
		\includegraphics[scale=0.1]{INDUCC_EJEMPLO1}
		\caption{}
		\label{INDUCC_EJEMPLO1}
	\end{figure}
	
	\textbf{Indica la expresión de la fem inducida en la espira cuando penetra en la región del campo magnético. Calcula el valor del campo si en el instante $t = \SI{2}{\second}$ la fem inducida es $\varepsilon = \SI{1,6}{\volt}$.}

	---

	Podemos calcular primero el flujo magnético que atraviesa la espira:

	\begin{equation}
		\phi = B \cdot S \cdot \cancelto{1}{\cos \varphi}
	\end{equation}

	Donde $S$ es el área de la espira y $\varphi$ es el ángulo, en este caso $\varphi = \SI{0}{\degree}$, entre el campo magnético y el plano de la espira. Se puede ver que la superficie es un triángulo rectángulo, por lo que su área es $S = \frac{\ell^2}{2}$. Asimismo, al tratarse de un MRU, podemos sustituir $\ell$ por $v \cdot t$, lo que nos queda:

	\begin{equation}
		\phi = B \cdot \frac{v^2 t^2}{2}
	\end{equation}

	Sustituyendo los datos que ya tenemos y derivando para obtener la fem inducida:

	\begin{equation}
		\phi (t) = B \frac{4t^2}{2} = 2Bt^2  \Longrightarrow \varepsilon = -\frac{d\phi}{dt} = -4Bt
	\end{equation}

	Sin embargo, en esta ecuación no se está teniendo en cuenta que la espira se encuentra a 2 metros de la región del campo magnético, por lo que el tiempo que usaremos será $t_{\textnormal{desfase}} = \frac{\SI{2}{\meter}}{\SI{2}{\meter\per\second}} = \SI{1}{\second}$. El tiempo que usaremos en la ecueción serán los dos segundos del enunciado menos el segundo de desfase $t = \SI{2}{\second} - \SI{1}{\second} = \SI{1}{\second}$. En el enunciado también nos decían que $\varepsilon = \SI{1,6}{\volt}$, por lo que:

	\begin{equation}
		\SI{1.6}{\volt} = -4B \cdot \SI{1}{\second} \Longrightarrow B = \frac{\SI{1.6}{\volt}}{4 \cdot \SI{1}{\second}} = \SI{0.4}{\tesla}
	\end{equation}

\end{mybox}





\clearpage

\subsection{Resumen de Fórmulas}

\begin{table}[H]
	\centering
	\renewcommand{\arraystretch}{1.5}
	\setlength{\tabcolsep}{6pt}
	\setlength{\aboverulesep}{0pt}
	\setlength{\belowrulesep}{0pt}
	\renewcommand{\tabularxcolumn}[1]{m{#1}}
	\rowcolors{2}{gray!10}{white}
	\begin{tabularx}{\textwidth}{
			>{\centering\arraybackslash}m{4.5cm}
			>{\centering\arraybackslash}X
			>{\centering\arraybackslash}m{3cm}}
		\toprule
		\rowcolor{gray!25}
		\textbf{Magnitud / Concepto} & \textbf{Fórmula} & \textbf{Unidades (SI)} \\
		\midrule
		
		\textsc{Flujo Magnético} & $\displaystyle \phi = \vec{B} \cdot \vec{S} \Longrightarrow \phi = B \cdot S \cdot \cos \varphi$ & \unit{\weber} \\
		
		\textsc{Ley de Faraday--Lenz} & $\displaystyle \varepsilon = -\frac{d\phi}{dt}$ & \unit{\volt} \\
		
		\bottomrule
	\end{tabularx}
	\caption{Formulario de Inducción Electromagnética}
	\label{INDUCC_TABLAFORMULARIO}
\end{table}



\end{document}